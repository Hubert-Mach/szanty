\section{Ballada o Jakubie}

Słowa: Aleksander Wojciechowski \\
Muzyka: Andrzej Borowski

\index{A w miasteczku, gdzie domy mają dachy spadziste}
\vspace{2em}
\begin{tabular}{@{}p{9cm}@{}l@{}}
A w miasteczku, gdzie domy mają dachy spadziste, & \bfseries A E \\
Czas wolno przez małe okna ciekł, & \bfseries D A \\
Mieszkał Jakub - jak mówią - spokojne chłopisko, & \bfseries A E \\
Niedaleko szumiał morza brzeg. & \bfseries D A \\
Ale kiedyś po nocy nieprzespanej do rana \\
Wziął Jakub węzełek, zamknął drzwi. \\
Ślady po nim na brzegu długa fala zabrała, \\
Teraz wiatrem tkane jego dni. \\
\end{tabular}

\vspace{1em}
\begin{tabular}{@{}p{9cm}@{}l@{}}
Ref.: Pokład pod stopą trzeszczy, & \bfseries D A \\
Maszty połamie sztorm. & \bfseries E fis cis \\
Po co się włóczysz po świecie, & \bfseries D A \\
Kiedy masz własny dom?          |bis & \bfseries E A \\
\end{tabular}

\vspace{1em}
Są porty, co mają kolorowe ulice, \\
Na których wesoły, tłumny gwar. \\
Chodził Jakub samotnie, dziwiło go wszystko, \\
Zamiast drzew tam kwitły gaje palm. \\
Kiedy statek odpływał, to się zmieniał krajobraz, \\
Za rufą leciały sznury mew. \\
Długo Jakub przy burcie na falochron spoglądał, \\
Aż nie zniknął z oczu jego brzeg. \\
(Aż nie zginął z oczu płaski brzeg). \\
\newpage

Posiwiał już Jakub, zieleń oczu spłowiała, \\
Łaskawie go witał każdy port. \\
Ale dalej na morze tęsknota go gnała, \\
Nie pamięta, gdzie jest jego dom. \\
Pewno gdzieś jeszcze pływa Jakub po morzach, \\
Zardzewiał w kieszeni jego klucz. \\
Pewnie drogi do domu odnaleźć nie może, \\
Zabrał mu ją marynarski trud. \\