\section{Itaka}

Słowa: A. Jarecki\\
Muzyka: Edward Pałasz

\index{Nie lubię, nie lubię wracać}
\vspace{2em}
\begin{tabular}{@{}p{7cm}@{}l@{}}
Nie lubię, nie lubię wracać, & \bfseries  a G a G \\
Hardą ręką kołatać do wrót. & \bfseries  C E7 \\
To gorsze niż stała praca, & \bfseries  a G a G \\
Taki jednorazowy powrót. & \bfseries  C E7 \\
Lecz jakie przestrzenie powietrza, & \bfseries  C E7 \\
Jakie przestrzenie wody & \bfseries C F E7 \\
Przemierzać będzie mnie trza, & \bfseries  a G a G \\
By nie wrócić jak Odys. & \bfseries  C E E7 \\
\end{tabular}

\vspace{1em}
\begin{tabular}{@{}p{7cm}@{}l@{}}
Ref.: Nie wrócę na Itakę & \bfseries  a d \\
Ocean tnąc na pół. & \bfseries  a D d \\
W przeciwną stronę sztagiem, & \bfseries  a d \\
Pod fale będę pruł. & \bfseries  a E7 \\
Ja o rodzinę nie dbam, & \bfseries  a d \\
Ani o kraju rząd. & \bfseries  a D d \\
Ja jeszcze z roczek, ze dwa, & \bfseries  a d \\
Popływać chcę pod prąd. & \bfseries  a E7 a \\
\end{tabular}

\vspace{1em}
Raz przyszedł do mnie Epikur, \\
Objął mocno, pokazał palcem - \\
Tam w dali słoik konfitur \\
I herbata, i ciasto z zakalcem. \\
Chleb pachnie spożywczym hurtem, \\
Ciałem wstrząsają dreszcze. \\
Hej tam - ster w lewo na burtę! \\
Dobranoc, popływam jeszcze. \\

Nie wracać wcale nie łatwo, \\
Droga długa, trudna i żmudna. \\
Korabiem płynę i tratwą, \\
Zmieniam łodzie, galery i czółna. \\
Gdy światem wstrząsa listopad, \\
Słyszę wołanie w eter. \\
To woła mnie Penelopa: \\
"Wróć - zrobiłam Ci sweter!" \\

Ktoś może lubi podróże \\
Przez bezdroża pustynne i obce, \\
Lecz ja pod żaglami służę, \\
Moim domem jest okręt. \\
Deszcz wymył stempel z paszportu, \\
Zdjęcie porwały sztormy. \\
Kto chce, niech wraca do portu, \\
Ja nie wrócę do normy. \\

Nie wrócę na Itakę \\
Ocean tnąc na pół. \\
W przeciwną stronę sztagiem, \\
Pod fale będę pruł. \\
A żonę, żołd i żakiet \\
Niech bierze sobie gach. \\
Nie wrócę na Itakę, \\
A jeśli... to we snach. \\