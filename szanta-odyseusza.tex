\section{Szanta Odyseusza}

Słowa R. Jóźwiak\\
Muzyka A. Chudź

\index{Co u syna Telemacha?}
\vspace{2em}
\begin{tabular}{@{}p{8cm}@{}l@{}}
Co u syna Telemacha? & \bfseries  d \\
Czy mu w głowie tylko flacha? & \bfseries  g \\
I jak chlapnie ze dwa łyki & \bfseries  g \\
Panny łapie za tuniki & \bfseries  A \\
Różnie bowiem jest gdy matka & \bfseries  d \\
Sama chowa nastolatka & \bfseries  g \\
Ale tatuś wraca z Troi & \bfseries  F C \\
I gdy trzeba dupę złoi & \bfseries  A d (g A d A D C) \\
\end{tabular}

\vspace{1em}
\begin{tabular}{@{}p{8cm}@{}l@{}}
Ref. \\
Brać za żagle się chłopaki & \bfseries  B \\
Pora płynąć do Itaki & \bfseries  F A \\
Raźno wciągać spinakera & \bfseries  d C F C F \\
Niech do przodu gna galera & \bfseries  C A7 d (g A d A d) \\
Całą płynę już dekadę \\
Może więc do domu zjadę  \\
Człek styrany jak w kieracie  \\
Pora sprawdzić co na chacie  \\
\end{tabular}
\vspace{1em}

Wiozę dla swej Penelopy \\
Cud biustonosz i rajstopy \\
Tyle nocy sama w wyrze \\
Trzeba dać po suwenirze \\
Może już wkurzona całkiem \\
Na mnie się zamierzy wałkiem \\
Gdzie się szlajał mąż gagatek \\
Przez ostatnie parę latek \\
\newpage

No a może Penelopa \\
Już innego trzyma chłopa \\
Tak że powrót mnie Odysa \\
Już jej całkowicie zwisa \\
Kiedy tylko już przyjadę \\
To po papie dam za zdradę \\
A z pomocą Telemacha \\
Powypruwam flaki z gacha \\


Potem człowiek sobie może \\
Powspominać przy amforze \\
Homer gdy mu się poleje \\
Zrobi z tego odyseję \\



