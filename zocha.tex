\section{Zocha}

Słowa i muzyka: Lucjusz Michał Kowalczyk\\

\index{Posłała mnie Zocha po zakupy w porcie}
\vspace{2em}
\begin{tabular}{@{}p{11cm}@{}l@{}}
Posłała mnie Zocha po zakupy w porcie, & \bfseries  D A D \\
Było to w Giżycku, a może w Sztynorcie. & \bfseries G D A \\
Dała mi pół stówy i na fanty torbę, & \bfseries  D A D \\
'Wracaj przed południem, bo Ci skuję mordę!'  <Tak mówiła!>  & \bfseries G D A D \\
\end{tabular}

\vspace{1em}
\begin{tabular}{@{}p{11cm}@{}l@{}}
Ref.: "Kup pół chleba i pół masła, i pół główki kapusty, & \bfseries  D G A D \\
Zapamiętaj, aby serek także był półtłusty." & \bfseries  D e A D \\
Ja to wszystko proszę państwa połowicznie załatwiłem, & \bfseries  D G A D \\
O kapuście zapomniałem i pół litra zakupiłem.  <A Ty pijaku!> & \bfseries  D e A D \\
\end{tabular}

\vspace{1em}
Posłała mnie Zocha po zakupy w porcie, \\
Było to w Giżycku, a może w Sztynorcie. \\
Pod Pułtuskiem Narew płynie od północy, \\
Puste półki ujrzały me piękne oczy. \\

\vspace{-1em}
Posłała mnie Zocha po zakupy w porcie, \\
Było to w Giżycku, a może w Sztynorcie. \\
Siedzę na półdupku, a tu wpół do trzeciej, \\
Pół głowy mi myśli, słońce pięknie świeci.  <Och jak pięknie!> \\

Posłała mnie Zocha po zakupy w porcie, \\
Było to w Giżycku, a może w Sztynorcie. \\
Wracając spotkałem kumpli, gdzieś w pół drogi, \\
Walniemy po grzdylu, bo nas bolą nogi.  <A co?> \\

Posłała mnie Zocha po zakupy w porcie, \\
Było to w Giżycku, a może w Sztynorcie. \\
Rozpoznałem Zochę o północy mgliście, \\
Gada coś półgębkiem o bezludnej wyspie.