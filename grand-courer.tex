\section{Grand Coureur}

sł. Janusz Sikorski\\
mel. trad. "Le Grand Coureur"

\index{Wielki korsarz "Grand Coureur"}
\vspace{2em}
\begin{tabular}{@{}p{10cm}@{}l@{}}
Wielki korsarz "Grand Coureur" - to był okręt krwawych łez. & \bfseries d\\
Kiedyśmy na morze szli, żeby tam Anglików bić, & \bfseries d\\
Zdrajca morze, nawet wiatr obrócił się przeciwko nam. & \bfseries F C d\\
\end{tabular}

\begin{tabular}{@{}p{10cm}@{}l@{}}
Razem chłopcy, hej! Wesoło, chłopcy, hej! & \bfseries d C d d C d\\
Razem chłopcy, hej! Wesoło, chłopcy, hej! & \bfseries d C d d C d\\
La, la, la, la, la, la, la, la, la, la, la, la, la, la, la. & \bfseries F C d C d\\
\end{tabular}

\vspace{1em}
Wypłynęliśmy z L'Orient, gładka fala, świeży wiatr. \\
Jeszcze bracie widać ląd, a już gnają nas do pomp. \\
Pierwszy podmuch złamał maszt, bo zgniły był cholerny wrak. \\

Razem chłopcy, hej! Wesoło, chłopcy, hej!... \\

Nową stengę cieśla dał, trzeba zapleść kilka want. \\
A tu znów cholerny bal, burtą stanąć trza do fal. \\
Hej tam! Ster prawo na burt! Odpalić mi ze wszystkich rur! \\

Razem chłopcy, hej! Wesoło, chłopcy, hej!... \\

Angol bardzo blisko był, lufy w rzędach miał jak kły, \\
Niósł po morzach nagłą śmierć, ale Francuz nie bał się. \\
Hej tam! Ster prawo na burt! Odpalić mi ze wszystkich rur! \\

Razem chłopcy, hej! Wesoło, chłopcy, hej!... \\

\newpage
On kulami pluł nam w nos, a my w niego cios za cios. \\
Hej, abordaż! Wczepiaj hak! Zaraz Angol będzie nasz! \\
A tu gruby korek mgły Angola nam na zawsze skrył. \\

Razem chłopcy, hej! Wesoło, chłopcy, hej!... \\

Tak minęło dwieście dni, zdobyliśmy pryzy trzy: \\
Pierwszy - wpół przegniły wrak, drugi - kapeć tyleż wart, \\
Trzeci - hulk, co woził gnój, z nim był też cholerny bój. \\

Razem chłopcy, hej! Wesoło, chłopcy, hej!... \\

Żeby nikt nie opadł z sił, doskonały prowiant był: \\
Żyły i zjełczały łój, zamiast wina - octu słój. \\
Suchar stary, ale był, choć w każdym robak biały żył. \\

Razem chłopcy, hej! Wesoło, chłopcy, hej!... \\

Srogo los po rufach lał, w porcie przyjdzie zdychać nam. \\
Dwieście dni i pusty trzos, pieski rejs, parszywy los. \\
Każdy zgubę widzi już i każdy szuka wyjścia dróg. \\

Razem chłopcy, hej! Wesoło, chłopcy, hej!... \\

Szyper jedną z armat wziął, skoczył z nią w przepastną toń. \\
Bosman ruszył w jego ślad, dzierżąc się kotwicy łap. \\
Ochmistrz, w wielkiej kłótni drań, pijany leń i złodziej dań. \\

Razem chłopcy, hej! Wesoło, chłopcy, hej!... \\
\newpage

Jaja były, kiedy kuk łyżką od śmierdzących zup \\
Sam do kotła wcisnął się, pierwszy raz był w kotle wieprz. \\
I odpłynął z wiatrem gdzieś, a niech go porwie piekła brzeg! \\

Razem chłopcy, hej! Wesoło, chłopcy, hej!... \\

Nawet autor pieśni tej ze zgryzoty skoczył z rej. \\
Huknął o kuchenny blat, prosto do kubryku wpadł. \\
No a skutek taki był, że okręt rozbił w drobny pył! \\

Razem chłopcy, hej! Wesoło, chłopcy, hej!... \\

Jeśli tej historii treść poruszyła kilka serc, \\
Dobrych manier nie brak wam, Hej, postawcie wina dzban! \\
Bo gdy się śpiewa, w gardle schnie, no a z wyschniętym gardłem źle. \\

Pijmy chłopcy, hej! Wesoło chłopcy, hej! \\
Razem chłopcy, hej! Wesoło chłopcy, hej! 