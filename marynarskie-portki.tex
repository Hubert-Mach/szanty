\section{Marynarskie portki}

Słowa: Andrzej Mendygrał\\
Muzyka:  trad.

\index{Z obory prosto wzięli ją, niewinny wiejski kwiat}
\vspace{2em}
\begin{tabular}{@{}p{9cm}@{}l@{}}
Z obory prosto wzięli ją, niewinny wiejski kwiat, & \bfseries  C F C \\
Nie miała jeszcze chłopa, choć szesnaście miała lat. & \bfseries  C G \\
W Hotelu "Księcia Jurka" kelnereczką była tam. & \bfseries C F C \\
Szefowa strzegła cnoty jej, ta najgroźniejsza z dam. & \bfseries  C G C \\
\end{tabular}

\vspace{1em}
\begin{tabular}{@{}p{9cm}@{}l@{}}
Ref.: Marynarskie portki opięte tu i tam, & \bfseries  G C \\
A w środku kawał chłopa - to marzenie każdej z dam. & \bfseries  F C G C \\
\end{tabular}

\vspace{1em}
Do miasta w marszu przybył Czterdziesty Drugi Pułk, \\
Łajdaków i bękartów banda, każdy z nich to zbój. \\
Burdele zatłoczyli, zapchali każdy bar, \\
Lecz kelnereczka wywinęła się z plugawych łap. \\

I dragoni Księcia Walii przybyli wkrótce też, \\
Ogiery słynne, wygłodzone wśród zimowych leż. \\
I w całym mieście słychać było kobiet płacz i krzyk, \\
Lecz hotelowej kelnereczki znów nie dorwał nikt. \\

Aż kiedyś przybył żeglarz, zwyczajny koł i cham, \\
Z dębowym sercem, krzepki w krzyżu, postrach cór i mam. \\
Przez siedem lat, czy więcej, żeglował gdzieś przez świat, \\
I widać było, o czym myślał przez te parę lat. \\

Poprosił ją o świecę, by drogę znaleźć mógł, \\
A potem o poduszkę i ciepły koc do nóg. \\
Powiedział, że nie dotknie jej, że jej nie przerwie snu, \\
Lecz gdyby mogła wleźć do wyra, cieplej będzie mu. \\

Niewinna kelnereczka bezmyślnie weszła tam, \\
A on się na niej znalazł w mig, ten brutal, drań i cham. \\
I wykorzystał do dna różnicę obu płci, \\
Na koniec rzekła tylko: "Chyba teraz ciepło ci?" \\

Gdy wczesnym rankiem żeglarz bezsilny z łóżka wstał, \\
Za kłopot i siniaki pieniążek srebrny dał. \\
Powiedział: "Jeśli córka będzie - w domu trzymaj ją, \\
A gdyby to był chłopiec - to na morze wyślij go. \\

Niech marynarskie portki, jak ja, na zadku ma, \\
Jak jego tatko, niech na reje, gdy mu każą, gna." \\