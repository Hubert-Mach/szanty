\section{Samantha}

Słowa i muzyka: Mirosław Kowalewski \\

\index{Ty nie jesteś kliprem sławnym}
\vspace{2em}
\begin{tabular}{@{}p{9cm}@{}l@{}}
Ty nie jesteś kliprem sławnym & \bfseries  a G a \\
"Cutty Sark" czy "Betty Lou", & \bfseries  a G a \\
W Pacyfiku portach gwarnych & \bfseries  F G a \\
Nie zahuczy w głowie rum. & \bfseries  F G a a \\
\end{tabular}

\vspace{1em}
Nie dla Ciebie są cyklony, \\
Hornu także nie opłyniesz, \\
W rejsie sławnym i szalonym, \\
W szancie starej nie zaginiesz. \\

\vspace{1em}
\begin{tabular}{@{}p{9cm}@{}l@{}}
Ref.: Hej "Samantha", hej "Samantha", & \bfseries  F F \\
Kiedy wiatr Ci gra na wantach, & \bfseries  C a \\
Gdy rysujesz wody taflę, & \bfseries  F d \\
Moje serce masz pod gaflem. & \bfseries  a a \\
Czasem ciężko prujesz wodę & \bfseries  C C \\
I twe żagle już nie nowe. & \bfseries  a a \\
Jesteś łajbą pełną wzruszeń, & \bfseries F G \\
Jesteś łajbą... co ma duszę. & \bfseries  a G a a \\
\end{tabular}

\vspace{1em}
Ale teraz wyznać pora, \\
Chociaż nie wiem czemu, psiakość, \\
Gdy Cię nie ma na jeziorach, \\
Na jeziorach pusto jakoś. \\

Gdy w wieczornej przyjdzie porze \\
Śpiewać zwrotki piosnki złudnej, \\
Gdy Cię nie ma na jeziorze, \\
To Mazury nie są cudne. \\

Czasem, kiedyś już zmęczona, \\
W chwili krótkiej przyjemności, \\
W złotych słońca stu ramionach \\
Ty wygrzewasz stare kości. \\

A gdy przyjdzie kres Twych dróg, \\
Nie zapłaczę na pogrzebie, \\
Wiem, że sprawi dobry Bóg, \\
Byś pływała dalej w niebie.