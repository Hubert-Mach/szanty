\section{Ballada o nieznanych lądach}

Słowa: Andrzej Nowicki\\
Muzyka: Wojciech Laskowski

\index{Kapitanie, ląd po zawietrznej!}
\vspace{2em}
\begin{tabular}{@{}p{7cm}@{}l@{}}
Kapitanie, ląd po zawietrznej! & \bfseries a e\\
Dobrze go widać stąd. & \bfseries F G C \textsuperscript{7+} \\
- Omińmy go. Bezpieczniej & \bfseries d a\\
Nieznany omijać ląd. & \bfseries F e a\\
\end{tabular}

\vspace{1em}
Kapitanie, może by przystań \\
Bezpieczną ląd ten nam dał? \\
- Dla nas na morzu noc mglista \\
I sztorm, i burza, i szkwał. \\

Kapitanie, tam w każdym barze \\
Zabawa, śmiechy i gin. \\
- Tutaj jest szczęście nasze, \\
Na morzu wzburzonym tym. \\

\begin{multicols}{2}
Kapitanie, dziewcząt dostatek \\
Może nas czeka drżąc? \\
- Naszą miłością ten statek, \\
To morze i ta noc. \\

Kapitanie, mgłą się znów okrył \\
Ten ląd daleki i znikł. \\
- Dobrze, że go nie odkrył, \\
Ani ja, ani ty, ani nikt. \\

Kapitanie, długoż musimy \\
Błąkać się w noce i dnie? \\
- Póki nam starczy siły, \\
Nim nie spoczniemy na dnie. \\

- Kurs w morze i naprzód pełna, \\
Odwróćcie od brzegu wzrok. \\
Na lądzie chwiejny, niepewny \\
Jest marynarza krok.
\end{multicols}