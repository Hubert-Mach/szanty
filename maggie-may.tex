\section{Maggie May}

Słowa: Krystyna Banaszewska-Rogacka\\
Muzyka:  trad.

\index{Co zdarzyło kiedyś się}
\vspace{2em}
\begin{tabular}{@{}p{9cm}@{}l@{}}
Co zdarzyło kiedyś się,  & \bfseries   C \\
Opowiedzieć dziś wam chcę,  & \bfseries   C\textsuperscript{7} F \\
Chociaż wielu z was historie takie zna.  & \bfseries   C G G\textsuperscript{7} \\
Głupio wyszło ale tu,  & \bfseries   G\textsuperscript{7} C \\
W starym porcie Liverpool,  & \bfseries   C\textsuperscript{7} F \\
Pierwszy raz ogołocono mnie do cna.  & \bfseries   G\textsuperscript{7} C \\
\end{tabular}

\vspace{1em}
\begin{tabular}{@{}p{9cm}@{}l@{}}
Ref.: O Maggie, Maggie May,  & \bfseries   G\textsuperscript{7} \\
Przepędzili Ciebie, hej,  & \bfseries   C \\
Do harówy na tasmański, dziki brzeg.  & \bfseries   G\textsuperscript{7} \\
Orżnęłaś wielu żeglarzy  & \bfseries   C \\
I wielu marynarzy,  & \bfseries   C\textsuperscript{7} F \\
Już nie będziesz po portach włóczyć się.  & \bfseries   G G\textsuperscript{7} C \\
\end{tabular}

\vspace{1em}
Odebrałem forsę swą \\
Za mój rejs do Seyleone, \\
A niemało tego było, wierzcie mi. \\
Nie wiedziałem o tym, że \\
Wkrótce za mnie weźmie się \\
Ta dziewczyna o imieniu Maggie May. \\

Nie wiedziałem o niej nic, \\
Kiedy w oko wpadła mi, \\
Jak fregata piękna była w sukni swej. \\
Nic się nie liczyło już, \\
Na nią więc obrałem kurs, \\
Celem moim była właśnie Maggie May. \\

Dialog kilka chwil trwał, lecz \\
Nie wyczułem w czym jest rzecz \\
I jak głupi dałem wziąć się jej na hol. \\
Szybko płynął czas i gin, \\
Byłem zachwycony tym, \\
Że w tawernie mogłem szaleć razem z nią. \\

Gdy mnie zbudził słońca blask, \\
Po dziewczynie mojej ślad \\
Wszelki zniknął, a wraz z nią mój cały trzos. \\
Smutny finał szaleństw tych, \\
Kubrak w zastaw musiał iść, \\
Buty także podzieliły jego los.