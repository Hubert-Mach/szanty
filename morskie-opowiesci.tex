\section{Morskie opowieści}

\index{Kiedy rum zaszumi w głowie}
\vspace{2em}
\begin{tabular}{@{}p{9cm}@{}l@{}}
Kiedy rum zaszumi w głowie, & \bfseries  d \\
Cały świat nabiera treści, & \bfseries  C \\
Wtedy chętnie słucha człowiek & \bfseries  d \\
Morskich opowieści. & \bfseries  F C d \\
\end{tabular}

\begin{multicols}{2}
Hej, ha! Kolejkę nalej! \\
Hej, ha! Kielichy wznieśmy! \\
To zrobi doskonale \\
Morskim opowieściom. \\

Kto chce, to niechaj słucha, \\
Kto nie chce, niech nie słucha, \\
Jak balsam są dla ucha \\
Morskie opowieści. \\

Kto chce, to niechaj wierzy, \\
Kto nie chce, niech nie wierzy, \\
Nam na tym nie zależy, \\
Więc wypijmy jeszcze. \\

Łajba to jest morski statek, \\
Sztorm to wiatr, co dmucha z gestem, \\
Cierpi kraj na niedostatek \\
Morskich opowieści. \\

Pływał raz marynarz, który \\
Żywił się wyłącznie pieprzem, \\
Sypał pieprz do konfitury \\
I do zupy mlecznej. \\

Może ktoś się bardzo zżymał, \\
Mówiąc, że to zdrożne wieści, \\
Ale to jest właśnie klimat \\
Morskich opowieści. \\

Kto chce, ten niechaj wierzy, \\
Kto nie chce, niech nie wierzy, \\
Nam na tym nie zależy, \\
Więc wypijmy jeszcze \\

Był na Lwowie raz praktykant, \\
Dwumetrowy wielki taki, \\
przy kotwicznym kabestanie, \\
Połamał handszpaki. \\

Kolumb odkrył Amerykę, \\
Kiedy ścigał się z Halikiem, \\
Indianie się zarzekali, \\
Że pierwszy był Halik. \\

O wyprawie wokół globu, \\
Też fałszywe są pogłoski, \\
Pierwszy żaden tam Magellan, \\
Tylko Baranowski. \\

Znałem kiedyś kapitana \\
wielki był wielbiciel sportu \\
Stawiał żagle na fujarze \\
I wychodził z portu \\

Rudy Bronek, kiedy popił, \\
Robił bardzo groźne miny, \\
W Sztokholmie skakał \\
do wody i gonił rekiny. \\

I choć rekin twarda sztuka, \\
Ale Bronek w wielkiej złości \\
Łapał gada za ogona \\
I mu łamał kości \\

Małysz siedzi już na belce, \\
Patrzy ze spokojem z góry, \\
Jak się wkurwi to doleci \\
Na same Mazury.  \\

Nelson, angielski Admirał, \\
Strzeliłby se w łeb i kwita, \\
Gdyby wiedział co dokonał, \\
Kloss, zwykły kapitan. \\

Kloss uciekał Niemcom zawsze \\
Nigdy jego nie złapali \\
A dorównać mu potrafił \\
Jedynie pies Szarik \\
\newcolumn

U sąsiadów w Ukrainie \\
W Czernobylu pierdyknęło, \\
Elektrownia wyleciała \\
Pół miasta zginęło \\

Bracia nic nie powiedzieli, \\
Myśleli, że się rozwieje, \\
Ale Szwedzi namierzyli, \\
Już się lugol leje. \\

Ref.: Hej, ha, lugola nalej, \\
Hej, ha, probówki wznieśmy, \\
Na chorobę popromienną, \\
Jest to lek najlepszy. \\

Ciągle straszą nas Reganem, \\
Rakietami i kosmosem, \\
A tu bracia robią prezent \\
Pod samiutkim nosem. \\

Kiedy tylko coś tam gruchnie, \\
Wschodni wiatr ku nam zawieje, \\
Wtedy każdy chętnie spieprza \\
W niedostępną knieję. \\

Na Podlasiu się rozeszła \\
Wieść, co przyszła prosto z pola:\\
Chodźcie ludzie do ośrodka, \\
Dają tam lugola. \\
\newcolumn

Krów na łące paść nie można, \\
Choć stężenie ciągle spada, \\
TASS agencja przecież wszędzie \\
Ciągle o tym gada. \\

Pierwszy Maja - Święto Pracy, \\
Na ulicach tłum wariuje, \\
A tu miłość od przyjaciół\\
Ciągle promieniuje.\\

Kiedy włosy mi wypadną\\
I biegunka mną zawładnie, \\
Co ja wtedy sobie myślę, \\
Tego nikt nie zgadnie.\\

W głowie coś mi strasznie łupie, \\
Skóra schodzi też płatami, \\
Lecz Ty miej to wszystko w dupie, \\
Śpiewaj razem z nami.\\

Kiedy zęby Ci wylecą, \\
Łysa pała Ci zostanie, \\
Wspomnisz Bracie dawne czasy, \\
Hej, lugola nalej!\\

Pij lugola, pij na zdrowie, \\
To się może przydać jeszcze, \\
Bo w Żarnowcu już budują, \\
Skończą za lat dwieście.\\
\newcolumn

Na Europę padł strach blady, \\
Co sprytniejszy schron buduje, \\
Płaszcz gumowy na się wkłada, \\
Mleka nie kupuje.\\

A ja przecież wdzięczny jestem, \\
W sercu się zabliźnia rana, \\
Że mi zginąć nie pozwolą\\
Od rakiet Reagana.\\

Pływał sobie kleryk, który\\
Myślał, że go dupa boli, \\
Patrzy, a tu arcybiskup\\
W koi go pierdoli.\\

Aby o arcybiskupie\\
Naród długo nie pamiętał.\\
Zaraz na pożarcie jemu\\
Dali dyrygenta\\

Jak spod Helu raz dmuchnęło, \\
Żagle zdarła moc nadludzka, \\
Patrzę - w koje mi przywiało\\
Nagą babkę z Pucka.\\

Niech drżą gitary struny, \\
Niech wiatr grzywacze pieści, \\
Gdy płyniemy pod banderą\\
Morskich opowieści.\\
\newcolumn

Był na "Lwowie" młody majtek, \\
Co nie ruchał przez rok chyba, \\
i wytryskiem swojej spermy\\
Zabił wieloryba\\

Raz bosmana rekin pożarł, \\
Lecz nie smućcie się kochani, \\
Bosman żyje, rekin umarł, \\
Zatruł się zbukami.\\

Kiedy bosman trypra złapał, \\
Obciął sobie własnym nożem, \\
A gdy rzucił go za burtę, \\
To wezbrało morze.\\

Znałem kiedyś Chinkę w barze, \\
Co śpiewała piosnki sprośne, \\
Gdy kimono swe rozdziała, \\
Cycki miała skośne\\

Pływał z nami raz szantymen, \\
śpiewał bardzo niskim basem, \\
W rękach zawsze miał gitarę, \\
Ster trzymał kutasem\\

Kiedy mocny szkwalik przywiał, \\
Ster mu w jaja tak przywalił, \\
Już nie basem, a sopranem, \\
Panny w porcie bawił.\\
\newcolumn

Gdy przy sterze dzisiaj siedzi, \\
Szanty sobie dalej śpiewa, \\
Bez akordów i gitary, \\
Tylko a'cappella.\\

Grant Kapitan z żoną pływał, \\
Nie dopatrzył raz załogi, \\
Odtąd ma bachorów kupę, \\
A na głowie rogi.\\

Znałem kiedyś kurtyzanę, \\
Co się zwała Ruda Brona, \\
W pół godziny obskoczyła\\
Eskadrę Nelsona.\\

Żyła raz w Londynie dziwka, \\
Co ją zwali Krwawa Bronka, \\
Bo jak zaciskała uda, \\
Obcinała członka.\\

I żadnemu żeglarzowi\\
Nie udało się jej dosiąść, \\
Bo dostawał opatrunek, \\
A ona korkociąg.\\

Ale trafił się marynarz, \\
Mieszkał podobno w Poczdamie, \\
Co drewnianą swą protezą\\
Zrobił kuku damie.\\
\newcolumn

Znałem raz murzynkę w Rio, \\
Co w miłości była śmiała, \\
Nie uwierzysz daję słowo, \\
Całkiem w poprzek miała.\\

Znałem kiedyś pannę śliczną.\\
Maszty stawiać uwielbiała, \\
Chłopa z łajbą pomyliła, \\
Lecz nie żałowała.\\

Znałem kiedyś marynarza, \\
Co miał pecha tak wielkiego, \\
Wsadził ptaszka między greting\\
I nie wyjął jego.\\

Żyła w Gdańsku cnotka Zocha, \\
Z każdym chciałaby się kochać, \\
Lecz stalową cnotę miała, \\
Rzewnie więc płakała.\\

Zośka dzięki swym przymiotom, \\
Podpuszczalska była wielce, \\
Wielu więc miało złamane, \\
Niekoniecznie serce.\\

Larsen choć był harpunnikiem, \\
Nie mógł Zośce przebić cnoty, \\
Choć jednym rzutem harpuna, \\
topił trzy U-Booty.\\
\newcolumn

Gdy kapitan zachorował, \\
Zrobiono mu lewatywę, \\
Wlano w niego galon wody, \\
Przez prezerwatywę.\\

Może biedak by wyzdrowiał, \\
Bo kuracja pierwsza klasa, \\
Ale kondom był dziurawy, \\
Dostał adidasa.\\

Kiedy smutno Ci na wachcie, \\
Chcesz rozerwać się troszeczkę, \\
Wkładasz granat między nogi\\
I ciągniesz zawleczkę, \\

Pływał raz marynarz który\\
Nosił w spodniach same dziury\\
Było widać przez te dziury\\
To co znoszą kury\\

Od Falklandu-śmy płynęli, \\
Doskonale brała ryba, \\
Mogłeś wędką wtedy złapać\\
Nawet wieloryba\\

Kiedy znudzą Ci się szanty, \\
I żegluga, i Mazury, \\
Olej twego kapitana, \\
I spierdalaj w góry.\\
\newcolumn

Znałem kiedyś marynarza, \\
Kochał piwo, no i tańce, \\
Jak się odlał to wypełniał\\
Śluzę na Guziance.\\

W Mikołajkach żył raz żeglarz, \\
Zjadł grochówki talerz cały, \\
Potem pierdząc z rufy jachtu\\
Przepłynął kanały.\\

W Amsterdamie była knajpa, \\
Która w herbie miała Gryfa, \\
Z której nikt nie wyszedł wcześniej, \\
Niż nie złapał syfa. \\

Eskimosi są weseli, \\
Myją moczem swoje żony, \\
Pędzą bimber z ryb popsutych\\
I zjadają wrony.\\

Grenlandczycy, lud gościnny, \\
Więc pracują gorączkowo, \\
Sprowadzają lód z Syberii, \\
Nie wiemy dla kogo.\\

Jak do Gdyni wracalismy, \\
Goła baba mi się śniła.\\
Gdy się rano obudziłem, \\
Dziura w burcie była.\\
\newcolumn

Na Yan Mayen wielka draka, \\
Opowiem, jak rzecz się miała, \\
Kiedy wzięto nas za desant, \\
Ludność się poddała.\\

Kiedy szliśmy przez Pacyfik, \\
Wiatr pozrywał wszystkie wanty, \\
Przytuliłem się do klopa\\
I śpiewałem szanty.\\

Kiedy szliśmy przez Pacyfik, \\
Była wtedy straszna flauta, \\
Wprost na łajbę nam się zjebał\\
Ruski kosmonauta.\\

Powiedziała mi dziewczyna\\
Żeby wodą wódkę popić.\\
Ja jej na to: "Idź do diabła, \\
Czy chcesz mnie utopić?"\\

Znałem raz pewnego majtka, \\
Kto nie wierzy, niech się śmieje, \\
Co swym chujem podczas wzwodu\\
Mógł zastąpić reje.\\

Pływał raz po morzu kucharz, \\
W rękach praktyk był Onana, \\
A załoga się dziwiła\\
Skąd w kawie śmietana.\\
\newcolumn

Pływał raz marynarz, który\\
Chuja miał jak trzy armaty\\
I wytryskiem swej giwery\\
Zatapiał fregaty. \\

Znałem raz pewnego kuka, \\
Co nie dupczył przez rok cały\\
I jak wsadził rybie w dupę, \\
To jej wyszły gały.\\

W dawnych czasach na żaglowcach\\
Żyły kozy tresowane, \\
Mogły one Ci zastąpić\\
Każdą kurtyzanę.\\

A gdy koza szła do kotła, \\
Bo czasami tak się zdarza, \\
To wtedy załoga cała\\
Jebała kucharza.\\

Pewien majtek miał papugę\\
Najsłynniejszą w całym świecie, \\
No bo była okrętową\\
Mistrzynią w minecie.\\

Za usługi tej papugi\\
Majtek pobierał dolara, \\
Nic dziwnego, w długim rejsie\\
Wzbogacił się zaraz.\\
\newcolumn

A dla kogo za papugę\\
Była to za duża kwota, \\
Mógł pożyczyć od bosmana\\
Szczerbatego kota.\\

Była sobie kiedyś łódka, \\
Gdzie się ciągle lała wódka, \\
Opierdalał się tam sternik\\
- Czerwony Październik.\\

Statek z portu już wypływa, \\
Każdy w kącie ściska babę, \\
Smutny tylko jest Warszawiak, \\
On kocha Warszawę.\\

Gdy do portu przybijemy, \\
Wszyscy się do knajpy pchają\\
I już po godzinie picia\\
Nogi wyciągają.\\

W knajpie "Pod ponurą małpą"\\
Rządzi jedna z takim zadkiem, \\
Że niejeden już nie wiedział, \\
Co zrobić z uszatkiem.\\

Był raz szyper kutra "Brzytwa", \\
Jedna myśl go wciąż trapiła, \\
Bo on myślał, że ma trypra, \\
A to była kiła.\\
\newcolumn

Bosman na gitarze grywa, \\
Idzie mu to całkiem klawo, \\
Czasem chwyty popierdoli, \\
Wszyscy biją brawo.\\

Sternik, stary pijanica, \\
Ciągle biega, wódki szuka, \\
Jeden dzień bez alkoholu\\
I wyzionie ducha.\\


Był raz sobie sternik, który\\
Na silniku ciągle pływał, \\
A że halsów nie rozróżniał, \\
Żagli nie używał.\\

Jak pływałem z chłopakami, \\
Cały miesiąc się nie myli, \\
Gdy poszli do "Tropicany", \\
To ich nie wpuścili.\\

Zwrotka ta jest ku przestrodze\\
Tych, co Włocha chcą w załodze.\\
Włosi zamiast śpiewać szanty\\
Haftują zza wanty.\\

Znacznie większa jest pociecha\\
Na jachcie z kucharza - Czecha.\\
Choć podaje knedle z ciastem, \\
Jest śmiesznym balastem.\\
\newcolumn

Znałem kiedyś marynarza, \\
Był najlepszy zaraz po mnie, \\
Gdyśmy w Świnoujściu chlali, \\
Nasrał mi na spodnie.\\

Kiedy pływaliśmy razem, \\
Nie było nam równych prawie, \\
Aż do czasu, kiedy poznał\\
Dziewczę na zabawie.\\

Zaraz stracił humor cały\\
Wciąż wpatrzony w nią jak ciele, \\
A najgorsze, że zapomniał, \\
Co to przyjaciele.\\

Tak to człeku w życiu bywa, \\
Gdy Ci baba na łeb wlezie, \\
Jak Cię wciśnie pod pantofel, \\
To tak, jakbyś nie żył. \\

W Węgorzewie mieszka dziewka, \\
Co każdemu się oddaje, \\
Z dołu Puszcza Amazońska, \\
Z góry Himalaje.\\

A jej siostra chodzi smutna, \\
Użyć też by sobie chciała, \\
Z góry Stepy Akermańskie, \\
A z dołu Sahara.\\
\newcolumn

Znałem w pewnym porcie dziewkę, \\
Była kiedyś niezła laska, \\
Odkąd walec ją przejechał, \\
Jest jak decha płaska.\\


Wiosłowanie trudna sztuka, \\
Więc do pniaka uwiązali.\\
Próbujemy drzewo wyrwać\\
Razem z korzeniami.\\

Gdy wiosłować już umiemy, \\
Nawet w miarę równo, składnie, \\
Trzeba żagle tak postawić\\
By nie spocząć na dnie.\\

Żagle białe już wciągnięte, \\
Mnóstwo było z tym zachodu, \\
Wreszcie stało się - płyniemy, \\
Lecz rufą do przodu.\\

Przy czwóreczce to DZ-ta\\
Jest łódeczką dla rajdowca, \\
Raz mieliśmy na bukszprycie\\
Głowę surfingowca.\\

Barka keją treningową, \\
Fakt pożałowania godny, \\
Gdy prędkości nie wytracisz, \\
Masz okręt podwodny.\\
\newcolumn

Więc kursanci na manewrach\\
Mieli z faktem niezłą walkę, \\
Raz miast żagle poluzować\\
Przyjebali w barkę.\\

Znałem pewnego żeglarza, \\
Pływaliśmy razem Tangiem, \\
Był tak silny, że na chuju\\
Umiał podnieść sztangę.\\

Ryży Czajnik, kiedy popił, \\
Robił bardzo głupie miny, \\
Czasem rzygał do kokpiku\\
I podpieprzał liny.\\

Przyszedł bosman do tawerny, \\
Piwo!!! wrzasnął gromkim głosem\\
I od razu dostał w zęby -\\
Nie powiedział "proszę".\\

Pływał kiedyś majtek jeden\\
Masochista-pederasta, \\
Swe klejnoty często prażył\\
W prodiżu do ciasta.\\

A, że właśnie na tym statku, \\
Był cholerny problem z cukrem, \\
Wszyscy ciągle się dziwili, \\
Skąd to ciasto z lukrem?\\
\newcolumn

Gdy w żołądku tłuką fale\\
I chęć życia znika wszelka, \\
Nie trać bracie animuszu -\\
Jest jeszcze butelka!\\

Jeśli myślisz, że piosenka\\
Trochę się nie trzyma kupy, \\
To układaj własne zwrotki, \\
Nie zawracaj dupy.\\

Kumpel nazwać swoją łajbę\\
Chciał tytułem jakiejś pieśni, \\
Ja mu na to - daj jej imię\\
"Morskie Opowieści". \\

W wokółziemski rejs ruszałem\\
Marząc o niejednym porcie, \\
Ale zawsze dryfowałem\\
Do "Zęzy" w Sztynorcie.\\

Choć dmuchało tylko "cztery", \\
Jechaliśmy wciąż "na kancie", \\
Bo stawiałem genakera\\
Na własnym palancie!\\

Gdy nam dno przebiła rafa\\
I nasz szyper wpadł w panikę, \\
Wtedy dziurę kuk załatał\\
Świeżym naleśnikiem.\\
\newcolumn

Gdy motorek diabli wzięli, \\
W sztormie poszły wszystkie szmaty, \\
To za spinakera robił\\
Biustonosz Agaty.\\

Dziś opowiem wam historię\\
Co zdarzyła się gdzieś w Rio\\
Bosman wódki nie chciał wypić\\
No i dostał w ryjo.\\

A że dzielnie się odgryzał\\
Cała gawiedź podkurwiona\\
Potem tylko wyszło na jaw\\
Wódka była chrzczona.\\

Bosman twarde jest chłopisko\\
Lepiej w drogę mu nie wchodzić\\
Pięciu gości już wynieśli\\
Szósty ledwie chodzi.\\

Barman co tę wódkę dawał\\
Długo biedak już nie pożył\\
Bo gdy bosman go zobaczył\\
W jaja mu przyłożył.\\

Po tej wielkiej bijatyce\\
Bosman w całym Rio znany\\
Tłum się kłębi dookoła\\
On jest podziwiany.\\
\newcolumn

Wszystkie dziwki są już jego\\
Każda darmo da swe ciało\\
Zrobią tu dla niego wszystko\\
Byle mu się chciało.\\

Każda chce by ją przeleciał\\
A on wie jak im dogodzić\\
Bo po jego wizytacji\\
Wciąż nie mogą chodzić.\\

Dzisiaj bosman uśmiechnięty\\
Znowu wódki się napije\\
Nikt mu w Rio nie podskoczy\\
Bo wtedy nie żyje.\\

Potem im zaśpiewał szantę\\
Znaną prawie w całym świecie\\
No a jaki był jej refren\\
Chyba sami wiecie.\\

Pływał z nami jeden majtek, \\
Co go nazywali Kajtek, \\
Dziś załoga przerażona, \\
Bo to była ona.\\

Tylko bosman o tym wiedział, \\
Nic nikomu nie powiedział, \\
Za to dupczył ją zawzięcie, \\
Takie miał zajęcie.\\
\newcolumn

Kuchnia nasza jest wspaniała, \\
Czterech już do morza wnieśli, \\
Pozostałych zaś latryna\\
Nie może pomieścić.\\

Znałem kiedyś dziewczę z Mazur, \\
Kochać bardzo się lubiła, \\
Gdy z góralem się przespała, \\
Zawału dostała.\\

Jedna dziewka była szybka, \\
Marynarzy pocieszała, \\
Jednak po szklaneczce whisky\\
Zaraz zasypiała.\\

Była w jednym porcie dziewka, \\
Co każdemu siebie dała, \\
Lecz ostatnio nie jest chętna, \\
Bo się zakochała.\\

Gdy zeszliśmy na nabrzeże, \\
W zapomnienie poszło morze, \\
Bo siedziało na polerze\\
Jakieś blond niebożę!\\

Popękały wszystkie spodnie, \\
Kiedy na nasz widok wstała, \\
Potem całe dwa tygodnie\\
Sufit oglądała!\\
\newcolumn

Kiedy już odpływaliśmy, \\
Dla uciechy i dla sportu\\
Wcisnęliśmy do bakisty\\
Nasze dziewczę z portu.\\

Choć się fajnie żeglowało, \\
Każdy szukał wnet doktora, \\
Bo się kurwa okazało -\\
Mewka była chora.\\

W dzień trapiła ją gorączka, \\
W nocy ją swędziła cipa;\\
Pewnie była to rzeżączka, \\
No bo skąd by grypa? \\

Coś na horyzoncie błyska, \\
Więc w przypływie euforii\\
"Pierwszy" wrzasnął, że to chyba\\
Brzegi Kalifornii!\\

To nie brzegi Sacramento, \\
Ino polska wieś Chałupy, \\
A te błyski, to nie złoto, \\
Tylko gołe dupy! \\

Chociaż była martwa cisza, \\
W dzień z Ogonków aż do Pisza\\
Przeszliśmy - lecz gorzko płacząc, \\
Bo ciągle burłacząc.\\
\newcolumn

Żreć będziecie dzisiaj z zęzy!\\
Krzyczę do załogi w porcie.\\
Całe szczęście dla załogi\\
To było w Sztynorcie.\\

W carskiej Rosji był krążownik, \\
Załoga się wygłupiała:\\
Gdy kapitan głośno pierdnął, \\
To strzelano z działa.\\

I przypadkiem tak się stało, \\
Chociaż za to nie dam głowy, \\
Dali sygnał do ataku\\
Na Pałac Zimowy.\\

Kuk gotował grochóweczkę, \\
Bo nic więcej nie potrafił, \\
Z Aurory śmierdziało z deczkę, \\
A Rosję szlag trafił..\\

\end{multicols}
