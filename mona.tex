\section{Mona}

Słowa: Anna Peszkowska\\
Muzyka:  trad.

\index{Był pięćdziesiąty dziewiąty rok}
\vspace{2em}
\begin{tabular}{@{}p{7cm}@{}l@{}}
Ref.: Był pięćdziesiąty dziewiąty rok, & \bfseries  a G a \\
Pamiętam ten grudniowy dzień, & \bfseries  a e \\
Gdy ośmiu mężczyzn zabrał sztorm & \bfseries a e a \\
Gdzieś w oceanu wieczny cień. & \bfseries  d e a \\
\end{tabular}

\vspace{1em}
\begin{tabular}{@{}p{7cm}@{}l@{}}
W grudniowy płaszcz okryta śmierć & \bfseries  a G a \\
Spod czarnych nieba zeszła chmur, & \bfseries a e \\
Przy brzegu konał smukły bryg, & \bfseries  a G a \\
Na pomoc "Mona" poszła mu. & \bfseries  d e a \\
\end{tabular}

\vspace{1em}
Gdy nadszedł sygnał, każdy z nich \\
Wpół dojedzonej strawy dzban \\
Porzucił, by na przystań biec, \\
By ruszyć w ten dziki z morzem tan. \\

A fale wściekle biły w brzeg, \\
Ryk morza tłumił chłopców krzyk, \\
"Mona" do brygu dzielnie szła, \\
Lecz brygu już nie widział nikt. \\

\vspace{-1em}
\begin{multicols}{2}
Na brzegu kobiet niemy szloch, \\
W ramiona ich nie wrócą już, \\
Gdy oceanu twarda pięść \\
Uderzy w ratowniczą łódź. \\
\newcolumn

I tylko krwawy słońca dysk \\
Schyliło już po ciężkim dniu, \\
Mrok okrył morze, niebo, brzeg, \\
Wiecznego całun ścieląc snu. \\

Wiem dobrze, że synowie ich \\
Też w morze pójdą, kiedy znów \\
Do oczu komuś zajrzy śmierć \\
I wezwie ratowniczą łódź.
\end{multicols}