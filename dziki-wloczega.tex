\section{Dziki włóczęga}

Słowa: Henryk ("Szkot") Czekała \\
Muzyka:  trad.

\index{Byłem dzikim włóczęgą przez tak wiele dni}
\vspace{2em}
\begin{tabular}{@{}p{9cm}@{}l@{}}
Byłem dzikim włóczęgą przez tak wiele dni, & \bfseries  D G \\
Przepuściłem pieniądze na dziwki i gin. & \bfseries  D A D \\
Dzisiaj wracam do domu, pełny złota mam trzos, & \bfseries  D G \\
I zapomnieć chcę wreszcie, jak podły był los. & \bfseries  D A D \\
\end{tabular}

\vspace{1em}
\begin{tabular}{@{}p{9cm}@{}l@{}}
Ref.: Już nie wrócę na morze & \bfseries  A \\
Nigdy więcej, o nie! & \bfseries | D G \\
Wreszcie koniec włóczęgi, & \bfseries  D G \\
Na pewno to wiem!          |bis  & \bfseries  D A D \\
\end{tabular}

\vspace{1em}
I poszedłem do baru, gdzie bywałem nie raz, \\
Powiedziałem barmance, że forsy mi brak. \\
Poprosiłem o kredyt, powiedziała: "Idź precz! \\
Mogę mieć tu stu takich na skinienie co dzień." \\

Gdy błysnąłem dziesiątką, to skoczyła jak kot \\
I butelkę najlepszą przysunęła pod nos. \\
Powiedziała zalotnie: "Co chcesz, mogę Ci dać." \\
Ja jej na to: "Ty flądro, spadaj, znam inny bar." \\

Gdy stanąłem przed domem, przez otwarte drzwi \\
Zobaczyłem rodziców - czy przebaczą mi? \\
Matka pierwsza spostrzegła, jak w sieni wciąż tkwię, \\
Zobaczyłem ich radość i przyrzekłem, że...  \\