\section{Emeryt}

Słowa: Tomasz Piórski\\
Muzyka: Ryszard Muzaj

\index{Leżysz wtulona w ramię i cichutko mruczysz przez sen}
\vspace{2em}
\begin{tabular}{@{}p{10cm}@{}l@{}}
Leżysz wtulona w ramię i cichutko mruczysz przez sen, & \bfseries  e h \\
Łóżko szerokie, ta pościel biała, za oknem prawie dzień. & \bfseries  G H\textsuperscript{7} \\
A jeszcze niedawno koja, a w niej cuchnący rybą koc, & \bfseries  e D \\
Fale bijące o pokład i bosmana zdarty głos. & \bfseries  G H\textsuperscript{7} \\
\end{tabular}

\vspace{1em}
\begin{tabular}{@{}p{10cm}@{}l@{}}
Ref.: To wszystko było - minęło, zostało tylko wspomnienie, & \bfseries  e D A e \\
Już nie poczuję wibracji pokładu, gdy kable grają. & \bfseries e D A e \\
Już tylko dom i ogródek, i tak aż do śmierci, \\
A przecież stare żaglowce po morzach jeszcze pływają. \\
\end{tabular}

\vspace{1em}
Nie gniewaj się kochana, że trudno ze mną żyć, \\
Że nie kupiłem mleka, zapomniałem gary zmyć. \\
Bo jeszcze niedawno statek mym drugim domem był, \\
Tam nie stało się w kolejkach, tam nie było miejsca dla złych. \\

Upłynie sporo czasu nim przyzwyczaję się, \\
Czterdzieści lat na morzu zamknięte w jeden dzień. \\
Skąd lekarz może wiedzieć, że za morzem tęskno mi, \\
Że duszę się na lądzie, że śni mi się pokład pełen ryb. \\

Wiem, masz do mnie żal - mieliśmy do przyjaciół iść, \\
Spotkałem kumpla z rejsu, on w morze wyszedł dziś. \\
Siedziałem potem na kei, ze łzami patrzyłem na port, \\
I dla mnie też przyjdzie taki dzień, że opuszczę go, a na razie... \\