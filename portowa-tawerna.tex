\section{Portowa tawerna}

Słowa i muzyka: Jerzy Porębski

\index{Powiedz mi, Stary, który to raz}
\vspace{2em}
\begin{tabular}{@{}p{6cm}@{}l@{}}
Powiedz mi, Stary, który to raz & \bfseries a\\
Dajemy w tej knajpie w gaz? & \bfseries G a\\
Który raz płyniemy w długi rejs, & \bfseries a\\
Nie ruszając tyłka z miejsc? & \bfseries G a\\
\end{tabular}

\vspace{1em}
\begin{tabular}{@{}p{6cm}@{}l@{}}
A knajpą kołysze całkiem bezpiecznie & \bfseries C\\
I prawie nam nie jest źle, & \bfseries d d\textsuperscript{7}\\
A biały żagiel zwisa bezwietrznie & \bfseries B\\
W tej tytoniowej mgle. & \bfseries E\textsuperscript{7}\\
\end{tabular}

\vspace{1em}
Czasem przysiądzie jakaś mewka \\
Gdzieś na relingu naszych spraw \\
I krzykniesz: "Daj butelkę Stefka!" \\
Zupełnie jak: "Grota staw!" \\

\begin{multicols}{2}
No, powiedz mi, Stary, ile jeszcze razy \\
Wpadniemy tutaj o morzu pomarzyć? \\
Czy to pretekstem jest, czy nałogiem, \\
Że wciąż chcemy ruszyć w drogę? \\
\newcolumn

A tuż przy kei stoją jachty wielkie, \\
Wpłynęły do portu jak ptaki do gniazd. \\
Wiosną żagle ubiorą, jak panny sukienkę \\
I innych zabiorą w świat. \\
A nam? Znów przybędzie lat. \\

Powiedz mi, Stary, który to raz \\
Dajemy w tej knajpie w gaz? \\
Który raz płyniemy w długi rejs, \\
Nie ruszając tyłka z miejsc?
\end{multicols}