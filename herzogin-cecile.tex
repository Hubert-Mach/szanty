\section{Herzogin Cecile}

Słowa: Henryk ("Szkot") Czekała\\
Muzyka: Ken Stephens

\index{W dół bałtyckich cieśnin}
\vspace{2em}
\begin{tabular}{@{}p{9cm}@{}l@{}}
W dół bałtyckich cieśnin, & \bfseries  e D e \\
Tam, gdzie boja wraku tkwi, & \bfseries  D e \\
- Leży na dnie dzielny bark "Herzogin Cecilie". & \bfseries  e D e D e D e \\
Z wiatrem w dół Kanału, & \bfseries  e D e \\
Choć bez pary zawsze szli, & \bfseries  D e \\
- Na dzielnej, dumnej i wspaniałej "Herzogin Cecilie", & \bfseries  e D e D e D e \\
"Herzogin Cecilie", & \bfseries  e D e \\
"Herzogin Cecilie",  & \bfseries  e D e \\
Na dzielnej, dumnej i wspaniałej "Herzogin Cecilie". & \bfseries  e D e D e D e \\
\end{tabular}

\vspace{1em}
W dół, tam gdzie jest Biskaj, \\
Na jedzenie zbrakło sił, \\
- Na dzielnej, dumnej i wspaniałej "Herzogin Cecilie", \\
Poprzez stare doldrums \\
Szybko z wiatrem przeszły dni, \\
- Na dzielnej... (itd.) \\

Ryczą już czterdziestki, \\
Już za nami setki mil, \\
Tuż obok Tasmanii, \\
Gdy uciekli z szkwału w mig. \\

Szlakami wielorybów, \\
Tam, gdzie Horn i rzadko sztil, \\
Na piękne Falklandy, \\
Gdzie albatros kończy dni. \\

\newpage
Choć z ciężkim ładunkiem, \\
Choć huragan trzymał ich, \\
Do Plymouth czas wchodzić, \\
Tam dowiemy się gdzie iść. \\

I biegiem do Hamstone, \\
Choć we mgle, wciąż naprzód szli. \\
Blisko jest już Sawmile Cove, \\
Skończone jego dni. \\
- Leży na dnie dzielny bark "Herzogin Cecilie". \\
"Herzogin Cecilie", \\
"Herzogin Cecilie", \\
Dziś leży na dnie dzielny bark "Herzogin Cecilie". \\