\section{Uderz w dzwon}

sł. Anna Peszkowska\\
mel. trad. "Strike the Bell"\\

\index{Po rufie, niczym lew, ponury, krąży ktoś}
\vspace{2em}
\begin{tabular}{@{}p{10cm}@{}l@{}}
Po rufie, niczym lew, ponury, krąży ktoś. & \bfseries G C\\
To jest nasz Second Mate, cholernie twardy gość. & \bfseries G A\textsuperscript{7} D\\
Na łbie ma wiele spraw, zapomniał widać, że & \bfseries G C\\
Marzymy, by uderzył w dzwon i nie spóźniał się! & \bfseries D D\textsuperscript{7} G\\
\end{tabular}

\begin{tabular}{@{}p{10cm}@{}l@{}}
Uderz w dzwon, Second Mate, walnij bracie w dzwon! & \bfseries D G\\
Patrz, idzie wiatr i już rozwiał brodę twą. & \bfseries C G D\\
Dziób znowu złapał kurs, a piasek w twoim szkle & \bfseries G C\\
Przesypał się, to wachty kres, więc: "Go! Walnijże!" & \bfseries D\textsuperscript{7} G\\
\end{tabular}

\vspace{1em}
A w dole szkafut drży, przy pompach praca wre. \\
Zwijają się tam chłopcy i myślą sobie, że \\
Mógłby im Second Mate darować parę chwil. \\
Prostować kark najwyższy czas, brakuje już sił. \\

Uderz w dzwon, Second Mate, walnij bracie w dzwon. \\
Patrz, stężał wiatr i już szarpie brodę twą. \\
Dziób kursu trzyma się, a piasek w twoim szkle \\
Przesypał się, to wachty kres, więc: "Go! Walnijże!" \\

Na oku Mały John już mocno w tyłek zmarzł. \\
Do koi ciepłej by - najchętniej nie sam - wlazł. \\
I chciałby ryknąć już, że światła palą się, \\
więc Second Mate uderzaj w dzwon, bo znów będzie źle! \\
\newpage

Uderz w dzwon, Second Mate, walnij bracie w dzwon! \\
Patrz, idzie szkwał i już zmoczył brodę twą. \\
Dziób ciężko ryje noc, a piasek w twoim szkle \\
Przesypał się, to wachty kres, więc: "Go! Walnijże!" \\

Na rufie dźwięczy szkło, to dzielny Stary nasz. \\
Co myśli, wiemy już: "Do żagli, kurza twarz!" \\
Nieprędko koja nas utuli w słodkim śnie. \\
Ach, Second Mate, ty sam to wiesz, że wciąż spóźniasz się. \\

Uderz w dzwon, Second Mate, walnij bracie w dzwon! \\
Patrz, masz już sztorm, co wytarga brodę twą. \\
Dziób ciężko ryje noc i choć piasek w twoim szkle \\
Przesypał się, to wachty kres, więc: "Go! Walnijże!" \\