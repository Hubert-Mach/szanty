\section{Szanta Oliwska}

Słowa: Tomasz Piórski\\
Muzyka:  trad.

\index{Był wiek XVII, rok dwudziesty siódmy}
\vspace{2em}
\begin{tabular}{@{}p{8cm}@{}l@{}}
Był wiek XVII, rok dwudziesty siódmy, & \bfseries F B C \\
Listopadowy wiatr zimny wiał. & \bfseries C\textsuperscript{7} F \\
Była niedziela i słońce wstawało, & \bfseries F B C \\
Na "Świętym Jerzym" sygnałem był strzał. & \bfseries F B C F \\
\end{tabular}

\vspace{1em}
\begin{tabular}{@{}p{8cm}@{}l@{}}
Ref.: Ma Anglia Nelsona i Francja Villeneuve'a, & \bfseries F B C \\
Medina-Sidonia w Hiszpanii żył. & \bfseries C\textsuperscript{7} F \\
A kto dziś pamięta Arenda Dickmanna? & \bfseries B C \\
To polski admirał, co z Szwedem się bił. & \bfseries F B C F \\
\end{tabular}

\vspace{1em}
Sześć szwedzkich okrętów na walkę czekało, \\
A każdy z nich większy od naszych dwóch. \\
Choć siły nierówne, bo dział było mało, \\
To w polskich załogach bojowy był duch. \\

\begin{multicols}{2}
Huknęła salwa, wiatr żagle złapały, \\
Błysnęły armaty z burtowych furt. \\
Dziesięć korabi do walki ruszało, \\
Od strony półwyspu stał szwedzki wróg. \\
\columnbreak

Błyskały armaty, gwizdały kartacze \\
I z trzaskiem okręty zwarły się dwa. \\
'Wodnik' przypuścił abordaż zuchwały, \\
Pogrążył się szwedzki galeon wśród fal. \\

'Tigern' i 'Solen' - okręty rozbite, \\
A reszta Szwedów uciekła gdzieś w dal. \\
Admirał Dickmann legł w boju zabity, \\
Lecz flota spod Orła zwycięstwo swe ma. \\
\end{multicols}