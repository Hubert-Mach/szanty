\section{Nie wiem skąd się wziął}

Słowa i muzyka: Jerzy Porębski

\vspace{2em}
\begin{tabular}{@{}p{10cm}@{}l@{}}
Nie wiem, skąd się wziął, bo nawet worka nie miał z sobą. & \bfseries a G\\
Nie wiem, skąd się wziął, tak niepozorną był osobą. & \bfseries F E\\
Nie wiem, skąd się wziął, gdy go ujrzałem w górnej koi, & \bfseries a G\\
Wyglądał tak, jak ten, co się własnego cienia boi. & \bfseries F E\\
\end{tabular}

\vspace{1em}
Potem przyszły dni, zwyczajne dni zejmana - \\
Bosman klął, a my czasem przez noc do rana. \\
I ładunek był zrobiony równo, wsad po wsadzie, \\
I wreszcie przyszły takie dni, że klar był na pokładzie. \\

Więc zapytałem go, jak to się zwykle dzieje, \\
Czy ma jakiś dom, czy przyjdzie ktoś na keję. \\
Popatrzył na mnie tak, że do dziś to w sercu mam: \\
"Ja, chłopie, na tym świecie jestem sam." - powiedział, \\
Albo jakoś tak - "Wśród ludzi tego świata jestem sam." \\

I głupio się zrobiło, bo niby razem, tak we dwoje, \\
Przeszliśmy długi rejs, przeżyli koja w koję, \\
Jedliśmy jeden chleb, śmierdzieli rybią łuską, \\
A teraz, gdy się kończy rejs, powiało pustką. \\

Zapytałem go jeszcze raz: "Czy schodzisz z tego wraka, \\
Czy jeszcze ze mną jeden rejs popłyniesz za rybaka?" \\
Popatrzył na mnie tak, jakby po raz pierwszy mnie zobaczył \\
I coś tam pod nosem zasobaczył. \\

Rzuciły losy nas po różnych kątach świata, \\
Minęło szereg długich lat, aż wreszcie los mi figla spłatał: \\
Zamustrowałem w rejs, gdzie on był kapitanem. \\
Myślałem: "Co też będzie, kiedy znów przed nim stanę?" \\

Pamiętam tamten zmierzch w salonie kapitana, \\
Ściskałem w ręku książkę, by mustrować za bosmana. \\
Spytałem go, czy wie, że ja go dobrze znam? \\
"Tak, teraz już nie będę sam" - powiedział, \\
Albo jakoś tak - "Tak, z Tobą już nie będę sam." \\

A ja nie wiem, skąd się wziął, bo nawet worka nie miał z sobą. \\
Nie wiem, skąd się wziął, tak niepozorną był osobą. \\
Nie wiem, skąd się wziął w tym kapitańskim salonie, \\
Kiedy wzruszony ja i on ściskaliśmy sobie dłonie. \\

Nie wiem, skąd się wziął... \\
