\section{Śmiały harpunnik}

Słowa: Marek Szurawski\\
Muzyka:  trad.

\index{Nasz "Diament" prawie gotów już}
\vspace{2em}
\begin{tabular}{@{}p{7cm}@{}l@{}}
Nasz "Diament" prawie gotów już, & \bfseries d a\\
W cieśninach nie ma kry, & \bfseries d a\\
Na kei piękne panny stoją, & \bfseries d a\\
W oczach błyszczą łzy.  & \bfseries g A7 d\\
\end{tabular}

Kapitan w niebo wlepia wzrok, \\
Ruszamy lada dzień, \\
Płyniemy tam, gdzie słońca blask \\
Nie mąci nocy cień. \\


\vspace{1em}
\begin{tabular}{@{}p{7cm}@{}l@{}}
Ref.: A więc krzycz: O-ho-ho! & \bfseries  d C d\\
Odwagę w sercu miej!  & \bfseries  C d\\
Wielorybów cielska groźne są,  & \bfseries  a g\\
Lecz dostaniemy je! & \bfseries  d a d\\
\end{tabular}

Hej, panno, powiedz po co łzy? \\
Nic nie zatrzyma mnie, \\
Bo prędzej w lodach kwiat zakwitnie, \\
Niż wycofam się. \\
No, nie płacz mała, wrócę tu, \\
Nasz los nie taki zły, \\
Bo da dukatów wór za tran \\
I wielorybie kły. \\
\newpage

Na deku Stary wąchał wiatr, \\
Lunetę w ręku miał. \\
Na łodziach, co zwisały już, \\
Z harpunem każdy stał. \\
I dmucha tu, i dmucha tam, \\
Ogromne stado wkrąg. \\
- Harpuny, wiosła, liny brać! \\
I ciągaj, brachu, ciąg! \\

I dla wieloryba już \\
Ostatni to dzień, \\
Bo śmiały harpunnik \\
Uderza weń! \\

| (szybciej) \\
Ref.: A więc krzycz: O-ho-ho! \\
Odwagę w sercu miej! \\
Wielorybów cielska groźne są, \\
Lecz dostaniemy je!
