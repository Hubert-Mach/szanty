\section{Hej, Mazury}

Słowa i muzyka: Lucjusz Michał Kowalczyk\\

\index{Już lato, więc elegancki porzucam świat}
\vspace{2em}
\begin{tabular}{@{}p{9cm}@{}l@{}}
Już lato, więc elegancki porzucam świat & \bfseries  D D \\
I czuję się jakbym miał znowu naście lat. & \bfseries  D A \\
I lecę tam, gdzie żeglarzy barwny tłok, & \bfseries  A A \\
Bo nie widziałem moich Mazur przecież przez cały rok. & \bfseries  A D \\
Już w Mikołajkach mnie czeka wrażeń moc, & \bfseries  D D \\
W tawernie lecą fajne szanty całą noc & \bfseries  D\textsuperscript{7} G \\
I każdy znajdzie tu dla siebie ciepły kąt, & \bfseries  G D \\
I nie ma takiej siły, która by mnie przegnała stąd. & \bfseries  A A \\
\end{tabular}

\vspace{1em}
\begin{tabular}{@{}p{9cm}@{}l@{}}
Ref.: Hej! Hej! He...ej! Mazury fajnie, że & \bfseries  G D \\
Mimo wszelkich przeciwności trzymacie się. & \bfseries A D D\textsuperscript{7} \\
Hej, hej, he-ej, Mazury kocham was, & \bfseries G D \\
Modre wody, złote piaski, zielony las. & \bfseries  A A \\
\end{tabular}

\vspace{1em}
A na biwaku też wcale nie jest źle, \\
Jasnym płomieniem już ognisko pali się. \\
W butelce mam na komary dobry lek, \\
No i dla siebie trochę czasu potrzebuje też człek. \\
Ach jak przyjemnie, gdy księżyc w trzcinach lśni \\
I ta gitara tam pod lasem tak cudownie brzmi. \\
Nic nie zakłóci tutaj mego snu, \\
Bo tego cholernego telefonu też nie ma tu. \\
\newpage

Do Węgorzewa! - Dziś taki zamiar mam, \\
Lecz pewnie stanę gdzieś na Dobskim, już ja siebie znam. \\
Do jedenastej w śpiworze lenię się, \\
Choć słonko dawno wstało i już leci wspaniały dzień. \\
Do chłodnej wody na golasa zrobię skok \\
I wszystkie rybki zawstydzone czmychną w bok. \\
Zutylizuję śmieci i pagajem machnę raz, \\
Bo właśnie teraz dla mnie wyjść na wodę nadszedł czas. \\

Postawię najpierw jeden żagiel, potem dwa, \\
I już wiaterek po jeziorku łajbę pcha. \\
Noga na rufie, plażowanie, pełny luz, \\
Bo nie istnieje przecież dla mnie żaden mus. \\
Na tej omedze, która ze mną ściga się, \\
Urocze trzy kobietki opalone śmieją się. \\
Chyba zluzuję jeszcze grota - wszak mam czas - \\
I tak w kanale znowu się spotkamy jeszcze raz. \\

Fajowo jest na Mazurach, każdy wie, \\
Zapraszam was, przekonajcie sami się. \\
Zapraszam w dobrą pogodę oraz złą, \\
Bo ukochane me Mazury to mój drugi dom. \\
Parara ram paj raram, taram daj, \\
Parara riri riri riri ri, \\
Ram dam daraj, raram taj, \\
Para rara, rara rara, ram tam tam... \\