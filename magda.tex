\section{Magda}

Słowa i muzyka: Andrzej Korycki\\

\index{Trójka wiała, jacht przy kei stał}
\vspace{2em}
\begin{tabular}{@{}p{9cm}@{}l@{}}
Trójka wiała, jacht przy kei stał, & \bfseries  d \\
Ale lało - mówię wam! & \bfseries  D\textsuperscript{7} g \\
Spać w ogóle się nie chciało, & \bfseries  g d \\
Mietek opowiadał nam. & \bfseries  A\textsuperscript{7} d \\
Nagle zbladł i rzecze: "Chłopcy, & \bfseries  d \\
Wam przestrogę teraz dam: & \bfseries  D\textsuperscript{7} g \\
Było to dwa dni po ślubie mym, & \bfseries  g d \\
Boże!... Ja to szczęście mam." & \bfseries  A\textsuperscript{7} d \\
\end{tabular}

\vspace{1em}
\begin{tabular}{@{}p{9cm}@{}l@{}}
Ref.: Magda, Magda, czemuś raz nie posłuchała, & \bfseries C\textsuperscript{7} F C\textsuperscript{7} F \\
Lecz jak zwykle ugasiłaś męża bunt? & \bfseries F D\textsuperscript{7} g \\
Powiedz Magda, czyś Ty zapomniała, & \bfseries C\textsuperscript{7} F B \\
Że na dwudziestu metrach trudno złapać grunt? & \bfseries F C\textsuperscript{7} F \\
\end{tabular}

\vspace{1em}
Mówi do mnie: "Mietuś, wiesz co? \\
Dziś będziem pływać łajbą twą." \\
Mówię: "Madziu, wiatru nie ma." \\
Patrzę, a ona wznosi dłoń. \\
Więc ja: "Dobrze Magda, ależ t...ak" \\
I na keję idę z nią. \\
A wtem jakieś złe przeczucie \\
Myśli mych zmąciło toń. \\

"Maczek" jednoosobowy był, \\
Ech, konstruktor płakałby. \\
No bo wszedłem - ja i Madzia ma - \\
To tak jak osoby trzy. \\
"Dmuchaj Mietuś, w żagle, dmuchaj, \\
Ja chcę pływać, pędem żyć!" \\
"Ty chcesz pływać?" - pomyślałem - \\
"To cholero przestań tyć!" \\

Nagle oko Magdy "mig i mryg". \\
Co jej? - zastanawiam się. \\
Instynktownie łapię pagaj, \\
Grzecznie pytam: "Co Ci jest?" \\
Żony drżąca ręka szuka mnie. \\
Myślę - O! Magdusi źle. \\
Nagle słyszę szept namiętny: \\
"Puść ten kij, obejmij mnie." \\

"Mak" się chwieje, ja tak myślę już - \\
Wlezie w kapok, czy też nie? \\
Wtem słoneczko zaświeciło, \\
Patrzę - No, opala się! \\
Nagle Magda zapragnęła ma \\
Zwilżyć w wodzie nóżkę swą. \\
Ja zdążyłem tylko krzyknąć: \\
"Nie wychylaj nogi, bo..." \\

Mietek przerwał... \\
Spojrzał po nas... \\
Otarł spływającą łzę \\
I na koniec tak powiedział: \\
"Jednak - nie zmieściła się..." \\