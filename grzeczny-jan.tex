\section{Grzeczny Jan}

Słowa: Mirosław Łebkowski, Stanisław Werner\\
Muzyka: R. Sielicki

\index{We wszystkich portach znali go}
\vspace{2em}
\begin{tabular}{@{}p{9cm}@{}l@{}}
We wszystkich portach znali go,  & \bfseries   d A\textsuperscript{7} \\
Przezwisko miał - Kulawy Joe,  & \bfseries   d A\textsuperscript{7} \\
A był to przecież chłop na schwał  & \bfseries   d g \\
I zdrowe obie nogi miał.  & \bfseries   C F A\textsuperscript{7} \\
Bo to jest zaszczyt dla korsarza,  & \bfseries   g \\
Gdy samo imię już przeraża.  & \bfseries   d A\textsuperscript{7} \\
We wszystkich portach znali go,  & \bfseries   d g \\
Przezwisko miał, przezwisko miał, przezwisko miał -  & \bfseries   A\textsuperscript{7} \\
- Kulawy Joe.  & \bfseries   A\textsuperscript{7} d \\
\end{tabular}

\vspace{1em}
Na jednej łajbie pływał z nim \\
Serdeczny kamrat - Ślepy Jim. \\
A, że niezgorsze oczy miał, \\
Z cholernym szczęściem w karty grał. \\
Bo to jest profit dla korsarza, \\
Gdy samo imię już przeraża. \\
Więc pływał Joe, a razem z nim \\
Serdeczny druh, serdeczny druh, serdeczny druh - \\
- Ten Ślepy Jim. \\

A trzeci to był Głuchy John, \\
Do tamtych dwóch pasował on, \\
Bo choć miał przytępiony słuch, \\
To słyszał każdy szept za dwóch. \\
I żyli zgodnie przyjaciele, \\
A każdy z małą skazą w ciele. \\
Kulawy Joe i Ślepy Jim, \\
Za nimi John, za nimi John, za nimi John - \\
- Szedł niby w dym. \\

W tawernie raz Jan suszył dzban, \\
A ściślej mówiąc - Grzeczny Jan. \\
Wtem wchodzi Joe i John, i Jim - \\
Do gustu Jan nie przypadł im. \\
Bo to jest hańba dla korsarza, \\
Marynarz o tak grzecznej twarzy. \\
Więc zaklął Joe i Jim, i John, \\
A każdy z nich, a każdy z nich, a każdy z nich - \\
- Na inny ton. \\

I mówią - "Tyś jest Grzeczny Jan, \\
To może z nami pójdziesz w tan?" \\
Więc Grzeczny Jan z uśmiechem wstał \\
I sfatygował kilka ciał. \\
Wynieśli Jima, Joe i Johna - \\
Dziś prawdę mówią ich imiona. \\
A Grzeczny Jan, że grzeczny był, \\
Spokojnie gin, spokojnie gin, spokojnie gin - \\
- Ze szklanki pił. \\

\vspace{2em}
\begin{tabular}{@{}p{9cm}@{}l@{}}
Ten Grzeczny Jan, ten Grzeczny Jan.  & \bfseries   A\textsuperscript{7} d A\textsuperscript{7} d \\
\end{tabular}