\section{Kongo River}

Słowa: Marek Szurawski\\
Muzyka: trad.

\index{Czy byłeś kiedyś tam nad Kongo River?}
\vspace{2em}
\begin{tabular}{@{}p{7cm}@{}l@{}}
Czy byłeś kiedyś tam nad Kongo River? & \bfseries D G\\
- Razem, hoo! & \bfseries e D G\\
Tam febra niezłe zbiera żniwo. & \bfseries D A D\\
- Razem weźmy go, hoo! & \bfseries A D\\
\end{tabular}

\vspace{1em}
\begin{tabular}{@{}p{7cm}@{}l@{}}
Brać się chłopcy, brać, a żywo! & \bfseries D G\\
- Razem, hoo! & \bfseries e D G\\
Cholerne brzegi Kongo River! & \bfseries D A D\\
- Razem weźmy go, hoo! & \bfseries A D\\
\end{tabular}

\vspace{1em}
\begin{multicols}{2}
Jankeski okręt w rzekę wpłynął, \\
Tam błyszczą reje barwą żywą. \\

Skąd wiesz, że okręt to jankeski? \\
Zbroczone krwią pokładu deski. \\

Ach, Kongo River, a potem Chiny, \\
W kabinach same sukinsyny. \\

Ach, powiedz, kto tam jest kapitanem? \\
John Mokra Śliwa - stary palant. \\

A kto tam jest pierwszym oficerem? \\
Sam Tander Jim, cwana cholera. \\
\columnbreak

A kogo tam mają za drugiego? \\
Sarapę - Johna Piekielnego. \\

A bosman - czarnuch zapleśniały, \\
Jack z Frisco liże mu sandały. \\

Cook lubi z chłopcem pobaraszkować \\
I nie ma czasu by gotować. \\

Jak myślisz, co mają na śniadanie? \\
Tę wodę, co miał Cook w kolanie. \\

Jak myślisz, co na obiad zjedzą? \\
Z handszpaka stek polany zęzą. \\

Nie zgadniesz, jaki tam jest ładunek \\
Z plugawych tawern sto dziewczynek.
\end{multicols}