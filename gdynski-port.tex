\section{Gdyński port}

Wykonawca: Muzaj Ryszard\\
Słowa i muzyka: Ryszard Muzaj

\index{Gdy nad gdyńskim portem zaświeci nam słońce}
\vspace{2em}
\begin{tabular}{@{}p{9cm}@{}l@{}}
Gdy nad gdyńskim portem zaświeci nam słońce  & \bfseries C G a e\\
Po długim, zimowym śnie, & \bfseries  F C F G\\
Rozkwitną znów białe żagle u rej & \bfseries  C G a e\\
Wietrze wiej im, o hej! & \bfseries  F G C\\
(Wietrze wiej, zimo hej!)
\end{tabular}

\vspace{1em}
Przytulna keja już męczy nas nudą \\
Powszednich, lądowych spraw... \\
Rozgorzał ogień nam w sercach, już czas \\
Hej, Żeglarze! Już czas! \\

\vspace{1em}
\begin{tabular}{@{}p{9cm}@{}l@{}}
Ref.: Kotwicę mokrą zawieśmy u burt, & \bfseries  F\textsuperscript{7+} C \\
Wiatr niechaj melodię gra! & \bfseries  F\textsuperscript{7+} C \\
Morze poniesie miarowy nasz wtór & \bfseries  F\textsuperscript{7+} C a \\
Żeglarskiej pieśni przez świat! & \bfseries  F G \\
\end{tabular}

\vspace{1em}
Już gdzieś za rufą pozostał nasz dom \\
Pełen ciepłych słów naszych żon, \\
Lecz nam nie pora spoglądać dziś wstecz  \\
Wiatr porywa nasz śpiew! \\

Gdy nad gdyńskim portem zaświeci nam słońce \\
Po długim, zimowym śnie, \\
Rozkwitną znów białe żagle u rej  \\
Wietrze wiej! Zimo hej!