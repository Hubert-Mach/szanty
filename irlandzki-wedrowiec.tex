\section{Irlandzki wędrowiec}

Słowa: Henryk ("Szkot") Czekała \\
Muzyka:  trad.

\index{I do przodu się rwał przez fale nasz ship}
\vspace{2em}
Raz! Dwa! Trzy! \\
\begin{tabular}{@{}p{9cm}@{}l@{}}
I do przodu się rwał przez fale nasz ship, & \bfseries   G C \\
Szesnasty to chyba był rok. & \bfseries   G D \\
Zły Przylądek Horn po nocach się śnił, & \bfseries   G C \\
Za rufą gdzieś został New York. & \bfseries   G D\textsuperscript{7} G \\
Był piękny, jak panna przed wyjściem na bal, & \bfseries   G D\textsuperscript{7} \\
Pulchniutki, krąglutki miał zadek. & \bfseries   G D\textsuperscript{7} \\
Dwadzieścia trzy maszty sterczały do chmur, & \bfseries   G C \\
Kto zgadnie, jak zwał się ten statek? & \bfseries   G D\textsuperscript{7} G \\
\end{tabular}

\vspace{1em}
Był tam Barney McGee, gdzieś z wybrzeża Lee \\
I Hogan, co w County miał swój bar. \\
Był tam John McGurk, no i Paddy Malone, \\
Co batem do pracy nas gnał. \\
Bill Casey, ten pijak i szuler jak nikt, \\
Niestety na pokład wlazł w Dover. \\
O reszcie nie wspomnę, bo zbraknie mi sił, \\
Brał wszystkich "Irlandzki Wędrowiec". \\

Ładunek zalegał od topu do dna, \\
Sam nie wiem, jak mógł zmieść się. \\
Prócz koni i kur, stu beczek bez dna, \\
Upchnięto po kątach co złe. \\
Te sześć milionów bel bawełny, co na dnie \\
Leżała w wodzie przez całe lata \\
I trzy miliony świń, i sześć milionów psów - \\
Sam diabeł nam figla tu spłatał. \\

Lecz stracił nasz statek swą drogę we mgle, \\
Przez sztorm, który przyniósł mu śmierć. \\
Z załogi tylko dwóch wytrwało po kres, \\
Nim złożył się lekko na dnie. \\
Zdradliwej skały ząb zakończył długi rejs, \\
Kapitan, podły tchórz, skoczył w morze, \\
Zostałem tylko ja, by wszystko to wam rzec... \\
Raz! Dwa! Trzy! \\
I jak zginął "Irlandzki Wędrowiec". \\