\section{Piętnastu chłopa}

Słowa i muzyka: Włodzimierz K. Głowacki\\

\index{Piętnastu chłopa na "Umrzyka Skrzyni"}
\vspace{2em}
\begin{tabular}{@{}p{7cm}@{}l@{}}
Piętnastu chłopa na "Umrzyka Skrzyni",  & \bfseries   a d a \\
- Ho-o-aj-ho! W butelce rum!  & \bfseries   H\textsuperscript{7} E E\textsuperscript{7} \\
Pij za zdrowie, resztę czart uczyni.  & \bfseries   a d a \\
- Ho-o-aj-ho! W butelce rum!  & \bfseries   H\textsuperscript{7} E E\textsuperscript{7} \\
\end{tabular}

\vspace{1em}
\begin{tabular}{@{}p{7cm}@{}l@{}}
Ref.: Z całej załogi jeden został zuch,  & \bfseries   a d a \\
Choć wypłynęło ich czterdziestu dwóch.  & \bfseries   d H\textsuperscript{7} E E\textsuperscript{7} \\
W butelce rum!  & \bfseries   a E E\textsuperscript{7} \\
W butelce rum!  & \bfseries   a E a \\
\end{tabular}

\vspace{1em}
Tom Evry dostał nożem po policzku, \\
A monsieur Tessain zadyndał na stryczku. \\

Nóż dostał w grdykę Francuz i padł trupem, \\
Grosz zaśniedziały był ich całym łupem. \\

Trup Bellamy'ego sczerniały i suchy, \\
Wisi pod Kingston, aż brzęczą łańcuchy. \\