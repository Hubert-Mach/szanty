\section{Mathew Anderson}

Słowa: Anna Peszkowska \\
Muzyka:  trad.

\index{Nazywam się Mathew Anderson}
\vspace{2em}
\begin{tabular}{@{}p{10.5cm}@{}l@{}}
Nazywam się Mathew Anderson, greenhornem byłem, gdy & \bfseries  e D C D e \\
Wlazłem na pokład jednego z tych, co hen, na południe szły, & \bfseries  e D C D e \\
A był to w życiu mym pierwszy rejs, poznałem więc morza smak, & \bfseries  G D G H\textsuperscript{7} \\
Ciężka robota tam szybko ugięła mój sztywny kark. & \bfseries  e D C D e \\
\end{tabular}

\vspace{1em}
\begin{tabular}{@{}p{10.5cm}@{}l@{}}
Ref.: Way-hey! Żegluj no! & \bfseries  G D \\
Strzeżcie córek swych w Hilo! & \bfseries  C D \\
Jankeski statek w zatokę wszedł, & \bfseries  G D G C \\
Potańcujemy tam sobie wnet! & \bfseries  e D e \\
\end{tabular}

\vspace{1em}
Harpunnik stary nauczył mnie przeliczać długie dni \\
Na krągłe kształty baryłek, co w gardziele ładowni szły. \\
A kiedy już dzielny statek nasz miał tranu pełny brzuch, \\
Odwrócił swój dziób i pochylił maszt, na Hilo Town obrał kurs. \\

Słońcem nas przyjął Hilo port, na brzegu dziewczyny już \\
Kolanem, ech, o kolano trą, a mężczyźni rwą czapki z głów. \\
A kiedy już nasz cuchnący tran na złoto przemienił się, \\
Lądowych uciech szedłem szukać sam, nie wiedząc, że zgubi to mnie. \\

Małej Jill w ręce wpadłem tam, miała oczy jak morza toń, \\
Podniosła raz rzęsy, a ciepło ich przypomniało stracony dom. \\
I dałem się jej niby dzieciak wieść - o swym ojcu mówiła mi, \\
Że pastorem jest i jak walczy ze złem, niby tarcza w tym mieście tkwi. \\
\newpage

Jak słońce, co wiosną roztapia kry, tak szybko sprawiła, że \\
Padłem na kolana i zgodziła się przysięgać na dobre i złe. \\
Nie zobaczę już Atlantyku kłów, gór mięsa płynących krwią, \\
Może czasem jeszcze przyniesie sztorm mej wolności piosenkę tą. \\
