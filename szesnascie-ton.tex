\section{Szesnaście ton}

Muzyka:  trad.\\

\index{Ktoś mówił, że z gliny ulepił mnie Pan}
\vspace{2em}
\begin{tabular}{@{}p{9cm}@{}l@{}}
Ktoś mówił, że z gliny ulepił mnie Pan, & \bfseries e H\textsuperscript{7} e H\textsuperscript{7} \\
Lecz przecież się składam z kości i krwi, & \bfseries e H\textsuperscript{7} e H\textsuperscript{7} \\
Z kości i krwi, i z jarzma na kark, & \bfseries e e\textsuperscript{7} a a\textsuperscript{7} \\
I pary rąk, pary silnych rąk. & \bfseries H\textsuperscript{7} e \\
\end{tabular}

\vspace{1em}
\begin{tabular}{@{}p{9cm}@{}l@{}}
Ref.: Co dzień szesnaście ton & \bfseries  e H\textsuperscript{7} \\
I co z tego mam? & \bfseries  e H\textsuperscript{7} \\
Tym więcej mam długów, & \bfseries  e H\textsuperscript{7} \\
Im więcej mam lat. & \bfseries  e H\textsuperscript{7} \\
Nie wołaj Święty Piotrze, & \bfseries  e e\textsuperscript{7} \\
Ja nie mogę przyjść, & \bfseries  a a\textsuperscript{7} \\
Bo duszę swoją oddałem za dług. & \bfseries  H\textsuperscript{7} e \\
\end{tabular}

\vspace{1em}
Gdy matka mnie rodziła, pochmurny był świt, \\
Podniosłem więc szuflę, poszedłem pod szyb. \\
Nadzorca mi rzekł - "Nie zbawi Cię Pan, \\
Załaduj co dzień po szesnaście ton." \\

Czort może dałby radę, a może i nie, \\
Szesnastu tonom podołać co dzień. \\
Szesnaście ton, szesnaście jak drut, \\
Co dzień nie da rady nawet i we dwóch. \\

Gdy kiedyś mnie spotkasz, lepiej z drogi mi zejdź, \\
Bo byli już tacy - nie pytaj, gdzie są. \\
Nie pytaj, gdzie są, bo zawsze jest ktoś, \\
Nie ten, to ów, co urządzi Cię. \\