\section{Znów popłynę na morze}

Słowa i muzyka: Jerzy Porębski

\index{Znów popłynę na morze kiwać się między falami}
\vspace{2em}
\begin{tabular}{@{}p{10cm}@{}l@{}}
Znów popłynę na morze kiwać się między falami, & \bfseries E\textsuperscript{7} a E\textsuperscript{7} a \\
znów popłynę na morze, pożegnam się z wami. & \bfseries G C H\textsuperscript{7} E\textsuperscript{7} \\
Słoną falę będę pił, słonym wiatrem będę żył, ale & \bfseries A\textsuperscript{7} d G C E\textsuperscript{7} \\
znów popłynę na morze z łajbą za pan brat. & \bfseries E\textsuperscript{7} a H\textsuperscript{7} E\textsuperscript{7} a \\
\end{tabular}

\vspace{1em}
\begin{tabular}{@{}p{10cm}@{}l@{}}
Nie wiem, nie wiem, może rejs ten będzie za długi,  & \bfseries E\textsuperscript{7} a \\
Nie wiem, nie wiem, słońce spotkam, czy zimne szarugi.  & \bfseries E\textsuperscript{7} a \\
Może z nóg mnie fala zwali, może się pokłonię fali, ale  & \bfseries A\textsuperscript{7} d G C E\textsuperscript{7} \\
\end{tabular}

\vspace{1em}
Znów popłynę na morze kąpać się z gwiazdami, \\
znów popłynę na morze, spotkam z kolegami. \\
W starych portach będę szperał, pewno spotkam przyjaciela, no bo \\
znów popłynę na morze z łajbą za pan brat.  \\

Nie wiem, nie wiem, może tęsknić będę za wami, \\
Nie wiem, nie wiem, może oczy zajdą mi łzami. \\
Może gdzieś na antypodach powiem, jednak życia szkoda, ale… \\

Znów popłynę na morze, jeszcze tego lata, \\
Wąchać różę wiatrów gdzieś na końcu świata. \\
Znów stare dżinsy włożę i pocudzołożę z morzem, no bo \\
znów popłynę w rejs do „zupełnie innych miejsc”.