\section{Kangur}

Słowa: Grzegorz ("Guru") Tyszkiewicz, Dariusz Ślewa \\
Muzyka:  trad.

\index{Szczęśliwy żywot wiodłem}
\vspace{2em}
\begin{tabular}{@{}p{10cm}@{}l@{}}
Szczęśliwy żywot wiodłem, kiedym przewoźnikiem był, & \bfseries  D G D G \\
Ale czart mi szepnął: 'W morze idź!' i rozbił wszystko w pył. & \bfseries  G e C D\textsuperscript{7} G \\
Bosmanem wnet zrobili mnie, krypa 'Kangur' zwała się, & \bfseries  D G D G \\
Babę fest znalazłem szybko, pomyślałem: 'Nie jest źle!' & \bfseries  G e C D\textsuperscript{7} G \\
\end{tabular}

\vspace{1em}
\begin{tabular}{@{}p{10cm}@{}l@{}}
Ref.: Gdy 'Kangur' ruszał z Bristol Town, & \bfseries  D \\
Przez myśl nie przeszło mi, & \bfseries  D G D G \\
Ile taka stara lafirynda może napsuć krwi. & \bfseries  G e C D G \\
\end{tabular}

\vspace{1em}
Cóż, lat troszeczkę miała, mówiła mi żem zuch, \\
Nie była starą panną, bo przede mną padło dwóch, \\
Lecz przyszło rzucić cumy nam, a że nie był ze mnie leń, \\
Osiemnaście pensów, jak w pysk dał, przynosił każdy dzień. \\

Pół roku diabli wzięli, lecz to głupstwo - mówię wam, \\
Pomyślałem: 'Wrócę, zejdę, no i już nie będę sam.' \\
Wróciłem, przekonałem się, żem głupi był i już, \\
Od pół roku żyła z bubkiem, który pływał wszerz, nie wzdłuż. \\

I tak marzenia wzięły w łeb, a Ty młody słuchaj też: \\
Jeśli dom chcesz mieć i żonę w nim, przez Kanał pływaj wszerz, \\
A dla mnie "Kangur" domem jest i na żonę późno już, \\
Przeżarte solą życie me, więc muszę pływać wzdłuż. \\