\section{Ludzie moi posłuchajcie}

Słowa i muzyka: Ryszard Muzaj\\

\index{Ludzie moi posłuchajcie, prawdę z życia powiem wam}
\vspace{2em}
\begin{tabular}{@{}p{10cm}@{}l@{}}
Ludzie moi posłuchajcie, prawdę z życia powiem wam & \bfseries E h \\
Morze żywi i bogaci, żeglarski niech żyje stan. & \bfseries A h E \\
Nigdzie nie doświadczysz wrażeń, których morze może dać, \\
O, mam na myśli sprawy proste, gdy po wachcie idziesz spać. \\
\end{tabular}

\vspace{1em}
\begin{tabular}{@{}p{10cm}@{}l@{}}
Ref.: Żegnaj "Home, sweet home", & \bfseries E h \\
Jesteś w rejsie i nie możesz zejść na ląd.  |bis & \bfseries h C D\textsuperscript{7+} (A E) \\
\end{tabular}

\vspace{1em}
Statek tańczy jak pijany, żółcią śmierdzi każdy kąt, \\
O, w mokrej kei sen przerwany, oto jedna z licznych klątw. \\
Wciskasz głowę w stęchłą pościel, w domu chciałbyś placek piec, \\
O, dobrze pić herbatę z rumem i po plaży sobie biec. \\

Chciałbyś wysiąść, masz już dosyć, mama zawsze rację ma, \\
O, dobrze, że tu Ewki nie ma, wstyd, bo cały toniesz w łzach, \\
Bo na lądzie jako-tako człek spokojny żywot ma, \\
Tutaj robisz za kanakę i kulszowa rwie Cię rwa. \\

A w porcie po skończonym rejsie trapem zbiegasz na swój ląd, \\
Skrzydła znowu Ci odrosły, mokry sztormiak rzucasz w kąt. \\
Sezon martwy się zaczyna, a pływanie aż za rok, \\
Teraz dumna twoja mina, Cape Horn'owca twardy wzrok. \\

Opowieści dziwne gadasz, Ewce już mętnieje wzrok, \\
Morskie węże, sztormy, glada no i ten szeroki krok. \\
I na szantach szanty śpiewasz głosem wilków morskich stu, \\
Jak Krutynią aż na Śniardwy pagajował dzielny zuch. \\

Sezon martwy w całej pełni, a pływanie aż za rok, \\
Teraz dumna twoja mina, Cape Horn'owca twardy wzrok. \\
I choć pędrak taki jesteś, dzieciuch, maluch, bąbel, brzdąc, \\
Kocham Cię, bo mimo wszystko znów za rok wypłyniesz stąd. \\

Ref.: I znów: "Home, sweet home", \\
Będziesz w rejsie \\
I daleko będzie ląd.  |x3 \\