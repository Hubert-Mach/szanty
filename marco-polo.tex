\section{Marco Polo}

Słowa: Sławomir Klupś \\
Muzyka:  trad.

\index{Nasz "Marco Polo" to dzielny ship}
\vspace{2em}
\begin{tabular}{@{}p{9cm}@{}l@{}}
Nasz "Marco Polo" to dzielny ship, & \bfseries  e G D e \\
Największe fale brał. & \bfseries  e G \\
W Australii będąc widziałem go, & \bfseries C e G D \\
Gdy w porcie przy kei stał. & \bfseries  e D e \\
\end{tabular}

\vspace{1em}
I urzekł mnie tak urodą swą, \\
Że zaciągnąłem się \\
I powiał wiatr, w dali zniknął ląd, \\
Mój dom i Australii brzeg. \\

\begin{tabular}{@{}p{9cm}@{}l@{}}
Ref.: "Marco Polo"  & \bfseries  e D C H\textsuperscript{7} \\
W królewskich liniach był. & \bfseries  e D e \\
"Marco Polo"  & \bfseries  e D C H\textsuperscript{7} \\
Tysiące przebył mil. & \bfseries  e D e \\
\end{tabular}

\vspace{1em}
\begin{multicols}{2}
Na jednej z wysp za korali sznur \\
Tubylec złoto dał \\
I poszli wszyscy w ten dziki kraj, \\
Bo złoto mieć każdy chciał. \\

I wielkie szczęście spotkało tych, \\
Co wyszli na ten brzeg, \\
Bo pełne złota ładownie są \\
I każdy bogaczem jest. \\

W powrotnej drodze tak szalał sztorm, \\
Że drzazgi poszły z rej, \\
A statek wciąż burtą wodę brał, \\
Do dna było coraz mniej. \\

Ładunek cały trza było nam \\
Do morza wrzucić tu, \\
Do lądu dojść i biedakiem być, \\
Ratować choć żywot swój.
\end{multicols}