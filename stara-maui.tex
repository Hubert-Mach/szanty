\section{Stara Maui}

Słowa: Henryk ("Szkot") Czekała\\
Muzyka:  trad.

\index{Mozolny, twardy i trudny jest}
\vspace{2em}
\begin{tabular}{@{}p{9cm}@{}l@{}}
Mozolny, twardy i trudny jest & \bfseries  d A d C \\
Nasz wielorybniczy znój, & \bfseries  d A\textsuperscript{7} d \\
Lecz nie przestraszy nas sztormów ryk & \bfseries  d A d C \\
I nie zlęknie groza burz. & \bfseries  d A\textsuperscript{7} d \\
Dziś powrotnym kursem wracamy już, & \bfseries F C A\textsuperscript{7} \\
Rejsu chyba to ostatnie dni  & \bfseries  d A\textsuperscript{7} \\
I każdy w sercu już chyba ma & \bfseries d A d C \\
Piękne panny ze starej Maui. & \bfseries  B A\textsuperscript{7} d \\
\end{tabular}

\vspace{1em}
\begin{tabular}{@{}p{9cm}@{}l@{}}
Ref.: Płyńmy w dół, do starej Maui, już czas, & \bfseries  C F C A\textsuperscript{7} \\
Płyńmy w dół, do starej Maui. & \bfseries  d A \\
Arktyki blask już pożegnać czas, & \bfseries  d A d C \\
Płyńmy w dół, do starej Maui. & \bfseries  B A\textsuperscript{7} d \\
\end{tabular}

\vspace{1em}
Z północnym sztormem już płynąć czas \\
Wśród lodowych, groźnych gór. \\
I dobrze wiemy, że nadszedł czas \\
Ujrzeć niebo z tropikalnych chmur. \\
Dziesięć długich miesięcy zostało gdzieś, \\
Wśród piekielnej, kamczackiej mgły. \\
Żegnamy już Arktyki blask \\
I płyniemy do starej Maui. \\
\newpage

Za sobą mamy już Diamond Head, \\
No i groźne stare Oahu. \\
Tam maszty i pokład na długo skuł \\
Wszechobecny, groźny lód. \\
Jak odrażająca i straszna jest \\
Biel Arktyki, tego nie wie nikt. \\
Za sobą mamy już setki mil, \\
Czas wziąć kurs do starej Maui. \\

Harpuny już odłożyć czas, \\
Starczy, dość już wielorybiej krwi. \\
Już pełne tranu beczki masz, \\
Płynne złoto sprzedasz w mig. \\
Za swój żywot psi, za trud i znój, \\
Kiedyś w niebie dostaniesz złoty tron. \\
O dzięki Ci Boże, że każdy mógł \\
Wrócić do rodzinnych stron. \\

Kotwica mocno już trzyma dno, \\
Wreszcie ujrzysz ukochany dom. \\
Przed nami główki portu już \\
I kościelny słychać dzwon. \\
A na lądzie uciech nas czeka sto, \\
Wnet zobaczysz dziatki swe. \\
Na spacer weźmiesz żonę swą \\
I zapomnisz wszystkie chwile złe