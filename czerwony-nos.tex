\section{Czerwony nos}

Słowa i muzyka: Lucjusz Michał Kowalczyk \\

\index{Ta historia, co wam powiem, jest całkiem prawdziwa}
\vspace{2em}
\begin{tabular}{@{}p{9cm}@{}l@{}}
Ta historia, co wam powiem, jest całkiem prawdziwa, & \bfseries  D A G A D \\
A dotyczy jegomościa, który ze mną pływał. & \bfseries  D A G A D \\
Facet nawet był do rzeczy, lecz miał małą wadę: & \bfseries  D A G A D \\
Nochal taki jak pomidor, chociaż lico blade. & \bfseries  D A G A D \\
\end{tabular}

\vspace{1em}
\begin{tabular}{@{}p{9cm}@{}l@{}}
Ref.: Wszyscy tak ubolewali, & \bfseries  D Fis\textsuperscript{7}  \\
Że go kopnął los, & \bfseries  h A \\
A on po prostu od gorzały & \bfseries  D G \\
Miał czerwony nos! & \bfseries  D A D \\
\end{tabular}

\vspace{1em}
Pewnej nocy o burtowych światłach dyskutował, \\
Że o mało jakiś rybol nas nie staranował, \\
A kapitan, człek wesoły, przy nim się nie ziewa, \\
Od tej pory stać mu kazał tam, gdzie burta lewa. \\

Pięć lat później spotkałem go na przedmieściu Gdyni, \\
Gdzie przy piwku młodej damie propozycje czynił. \\
Gapiłem się zadziwiony, aż mnie wzięła czkawka, \\
Bo malutki nosek miała, taki jak truskawka. \\

Mając wzgląd na barwy twarzy, chłop miał wielkie wzięcie \\
Przy państwowych różnych zlotach, albo ważnym święcie. \\
Gdy rocznice nastawały, lub też Pierwszy Maja, \\
On się zawsze kręcił z przodu... chyba tak dla jaja!