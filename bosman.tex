\section{Bosman}

\index{Na pokładzie od rana}
\vspace{2em}
\begin{tabular}{@{}p{7cm}@{}l@{}}
Na pokładzie od rana & \bfseries a C \\
Ciągle słychać bosmana, & \bfseries a A\textsuperscript{7}  \\
Bez potrzeby cholernie się drze. & \bfseries d a \\
Choćbyś ręce poranił, & \bfseries A\textsuperscript{7}  d \\
Bosman zawsze Cię zgani & \bfseries a \\
I powiada - "Zrobione jest źle!" & \bfseries E7\textsuperscript{7}  a \\
\end{tabular}

\vspace{1em}
"Jeszcze raz czyścić działo, \\
Cóż wam chłopcy się stało? \\
Jak do żarcia, to każdy się rwie. \\
To nie balia, nie niecka, \\
Trzeba wiedzieć od dziecka, \\
Że to okręt wojenny R.P." \\

\begin{multicols}{2}
Ale czasem się zdarzy, \\
Że się bosman rozmarzy  \\
Każdy bosman uczucie to zna. \\
Gdy go wtedy poprosisz, \\
Swą harmonię przynosi, \\
Siada w kącie na rufie i gra. \\

Opowiada o morzach, \\
O bezkresnych przestworzach \\
I o walkach, co przeżył on sam. \\
O dziewczętach z Bombaju, \\
Co namiętnie kochają \\
I całują tak mocno do krwi. \\

A gdy spytasz go tylko, \\
O czym marzył przed chwilką, \\
Czemu nagle pojaśniał mu wzrok? \\
Mówi: "W Gdyni, w Orłowie, \\
Będę chodził na głowie, \\
Tak mi przypadł do serca ten port." \\

Bosman skończył, wiatr ścicha, \\
Aż tu nagle, u licha! \\
Pojaśniało coś nagle we mgle. \\
Poznał bosman, jak cała \\
Polska w blaskach wstawała, \\
Na pokładzie okrętu R.P.
\end{multicols}