\section{Przechyły}

Słowa i muzyka: Paweł Orkisz\\

\index{Pierwszy raz przy pełnym takielunku}
\vspace{2em}
\begin{tabular}{@{}p{9cm}@{}l@{}}
Pierwszy raz przy pełnym takielunku & \bfseries  e D e \\
Biorę ster i trzymam kurs na wiatr. & \bfseries  e D e \\
I jest jak przy pierwszym pocałunku, & \bfseries  a H\textsuperscript{7} e \\
W ustach sól, gorącej wody smak. & \bfseries a H\textsuperscript{7} e \\
\end{tabular}

\vspace{1em}
\begin{tabular}{@{}p{9cm}@{}l@{}}
Ref.: O-ho-ho, przechyły i przechyły, & \bfseries  a H\textsuperscript{7} e \\
O-ho-ho, za falą fala mknie, & \bfseries  a H\textsuperscript{7} e \\
O-ho-ho, trzymajcie się dziewczyny - za liny! & \bfseries  a H\textsuperscript{7} e \\
Ale wiatr, ósemka chyba dmie! & \bfseries  a H\textsuperscript{7} e \\
\end{tabular}

\vspace{1em}
Zwrot przez sztag - o'key, zaraz zrobię, \\
Słyszę jak kapitan cicho klnie. \\
Gubię wiatr i zamiast w niego dziobem, \\
To on mnie od tyłu - kumple w śmiech. \\

Hej, Ty tam, za burtę wychylony, \\
Tu naprawdę się nie ma z czego śmiać. \\
Cicho siedź i lepiej proś Neptuna, \\
Żeby coś nie spadło Ci na kark. \\

Krople mgły, w tęczowych kropel pyle \\
Tańczy jacht, po deskach spływa dzień. \\
Jutro znów wypłynę, bo odkryłem, \\
Że wciąż brzmi żeglarska stara pieśń.