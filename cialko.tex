\section{Ciałko}

sł. i muz. Jerzy Porębski\\


\vspace{2em}
\begin{tabular}{@{}p{10cm}@{}l@{}}
Było to w maju słonko skwierczało & \bfseries C\\
A z podkoszulkiem sklejone ciało & \bfseries C\textsuperscript{7}\\
Czegoś by chciało & \bfseries F\\
Czegoś by chciało & \bfseries C\\
\end{tabular}

\begin{tabular}{@{}p{10cm}@{}l@{}}
Więc ja mu naprzeciw no tak dla ugody & \bfseries C\\
Na łeb wylałem wiaderko wody & \bfseries C\textsuperscript{7}\\
A ono mimo że wychłódniało & \bfseries F\\
Znów czegoś chciało & \bfseries C\\
\end{tabular}

\begin{tabular}{@{}p{10cm}@{}l@{}}
No może za dużo coś wczoraj zjadło & \bfseries E\textsuperscript{7}\\
Może uwiera go w boku sadło & \bfseries a\\
Może by chciało iść do lekarza & \bfseries D\textsuperscript{7}\\
Nie nie powtarza Nie nie powtarza & \bfseries GG\textsuperscript{7}\\
\end{tabular}

\vspace{1em}
Na dziób poszedłem z dala od ludzi \\
Pytam to cielsko No czemu się nudzi \\
Może do wachy chce w dół do rekina \\
Czy odpowiada mu barwa sina \\

Czemu mi robi wstyd przed kumplami \\
Oni nie mają takich trabli z ciałami \\
Może by chciało przejrzeć świerszczyka \\
Sprawdzić czy jeszcze zegarek cyka \\

Na nic sposobów wszelkich próbuję \\
Cielsku wyraźnie czegoś brakuje \\
To się zaziębi to się zapoci \\
To zechce śledzi to znów łakoci \\

Więc myślę sobie załatwię tego drania \\
Nastawiam budzik pośrodku spania \\
I ciemną nocą to cielsko budzę \\
Patrzę a ono takie no jakby cudze \\

Wierci się kręci coś kryje na dnie \\
Lecz ja każdym nerwem czuję dokładnie \\
Że nie chce wyznać mnie byle komu \\
Że bardzo chce już wrócić do domu \\

Ludzie Ja morze kocham Ja morze szanuję \\
Ja się na morzu bardzo dobrze czuję \\
Lecz wożę z sobą tę kupę złomu \\
Co nic tylko marzy by wrócić do domu \\

Więc pytam ja Ciebie mój Profesorze \\
Kto w takiej wojnie wygrać może \\
Gdy cielsko do powrotu zmusza \\
A znów na morze ciągnie dusza \\

Wygra entropia kancer wszechświata \\
Wojnę co trwa już świetlne lata \\
A mnie dwadzieścia lat przeleciało \\
Ja i to ciało \\

No więc było to w lutym słonko skwierczało \\
A z podkoszulkiem sklejone ciało \\
Znów czegoś chciało \\
Znów czegoś chciało