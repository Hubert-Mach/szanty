\section{Szekla}

Słowa i muzyka: Lucjusz Michał Kowalczyk \\

\index{Siedzę sobie tu przy maszcie od stu lat}
\vspace{2em}
\begin{tabular}{@{}p{9cm}@{}l@{}}
Siedzę sobie tu przy maszcie od stu lat, & \bfseries  C G\textsuperscript{7} C \\
Już niejeden chciał mnie ruszyć dzielny chwat. & \bfseries  C G\textsuperscript{7} C \\
Próbowali, oglądali, jednak rady mi nie dali, & \bfseries  F C D\textsuperscript{7} G \\
Klęli mnie na cały głos, a sukces był o włos. & \bfseries F C D\textsuperscript{7} G G\textsuperscript{7} \\
\end{tabular}

\vspace{1em}
\begin{tabular}{@{}p{9cm}@{}l@{}}
Ref.: No, bo ja jestem taka mała, & \bfseries  C G\textsuperscript{7} C C\textsuperscript{7} \\
Szekla zardzewiała, & \bfseries  F C \\
Chociaż życie mi poświęcisz, & \bfseries F C \\
To mnie nigdy nie odkręcisz. & \bfseries  D\textsuperscript{7} G \\
Choćbyś kręcił razy sto, & \bfseries  F C \\
Nie uda Ci się to! & \bfseries  D\textsuperscript{7} G C \\
\end{tabular}

\vspace{1em}
Widok dobry z tego miejsca wkoło mam, \\
Postukuję w maszt wesoło - co mi tam! \\
Kiedy fala mnie spłukuje, nieco jeszcze skoroduję, \\
Jednak tym nie martwię się, wiatr wysuszy mnie! \\

Rzekł raz Stary do bosmana: 'Słuchaj Zdziś, \\
Sprawa ma być rozwiązana jeszcze dziś.'  <mów tak dłużej!> \\
Bosman złamał trzy brzeszczoty, łeb mu zlały siódme poty, \\
A kapitan widząc to, ręką machnął... O! \\

Raz osiłek z wielkim młotem na mnie wpadł, \\
Tłukł mnie z furią i łoskotem, w końcu zbladł. \\
Taką mi urządził mękę, bo mu roz... ciachałam rękę, \\
Jeszcze mowę do mnie miał, brzydkie słowa znał. \\
\newpage

Młody żeglarz, co zębami miażdżył szkło, \\
Wziął mnie w gębę, jak kleszczami... No i co!? \\
I z uśmiechu jak marzenie pozostało mu wspomnienie. \\
Dziś omija mnie jak wesz, mało mówi też. \\

Okularnik z wielką głową tak, jak dzban, \\
Chciał mnie zniszczyć naukowo - mówię wam! \\
Trzy godziny medytował, mierzył, sprawdzał, kalkulował, \\
Wreszcie splunął na mnie: Tfuu! Danych zabrakło mu. \\

Ref.: No, bo ja jestem taka mała, \\
Szekla zardzewiała, \\
Chociaż życie mi poświęcisz, \\
To mnie nigdy nie odkręcisz. \\
Choćbyś kręcił razy sto, \\
Nie uda Ci się to! \\
Nie uda Ci się to! \\
Nie uda... Ci się... toooo...!