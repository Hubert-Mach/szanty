\section{Gruby}

Słowa i muzyka: Lucjusz Michał Kowalczyk \\

\index{W pewnym porcie, już sam nie wiem gdzie}
\vspace{2em}
\begin{tabular}{@{}p{9cm}@{}l@{}}
W pewnym porcie, już sam nie wiem gdzie, & \bfseries  e h \\
Ta historia wydarzyła się, & \bfseries  D A \\
W nędznym barze jakiś starzec ścierał kurz. & \bfseries  a G H\textsuperscript{7} e H\textsuperscript{7} \\
Nagle cisza zabiła gwar i śmiech, \\
Kiedy w drzwiach stanęło drabów trzech, \\
Oniemiałem, bo ich znałem wcześniej już. \\
Jeden z nich przekrzywiony miał łeb, \\
Mordę jak z cuchnącą rybą sklep, \\
Nic dobrego, biło z niego samo zło. \\
Taki był on, ech, psia jego mać, \\
Ludzi trącał, ciągle chciał się prać, \\
Uciekali, bo się bali wszyscy go. & \bfseries  a G H\textsuperscript{7} e \\
\end{tabular}

\vspace{1em}
\begin{tabular}{@{}p{9cm}@{}l@{}}
Ref.: Hej Ty, Gruby, nalej jeszcze raz! & \bfseries  G D H\textsuperscript{7} e \\
Biegną dni jak fale, pędzi z wiatrem czas. & \bfseries  G D H\textsuperscript{7} e \\
Hej ty, Gruby, powiedz, jak to jest, & \bfseries G D H\textsuperscript{7} e \\
Że spokojny człowiek jest ścigany jak pies? & \bfseries  G D H\textsuperscript{7} \\
Jak piee...eee...es! & \bfseries H\textsuperscript{7} \\
Jak piee...eee...es! & \bfseries  H\textsuperscript{7} \\
\end{tabular}

\newpage
Stali tak, rozglądali się w krąg, \\
Z trudem ukrywałem drżenie rąk, \\
Przeczuwałem, już widziałem własny zgon. \\
Na nic to, że spuściłem wzrok, \\
Skierowali jednak do mnie krok, \\
Zrozumiałem - teraz ja albo on. \\
Nie zwlekałem, stłukłem kufel o stół, \\
Szklane zęby wbiłem mu w brzucha dół, \\
Odstąpili ci, co byli przy nim tuż. \\
Skręcił się ten portowy gad, \\
I jak stał, tak z jękiem padł na blat, \\
Uciekałem, dzień witałem w morzu już. \\

Pytasz się, co on do mnie miał, \\
Że mnie tak zawzięcie dopaść chciał? \\
Z wielkim fartem kiedyś w karty ze mną grał. \\
Wkurzał mnie jego cygar smród, \\
Ograł mnie, no to łajbę spaliłem mu. \\
Może o to taki żal do mnie miał? \\
Teraz znów jego kumpli dwóch, \\
Chcą mi oddać to, co ja dałem mu. \\
W niepewności, w nieufności biegną dni. \\
Chociaż czasem jest gorzej niż źle, \\
To na ląd nie zejdę już nigdy, o nie! \\
Podła zemsta wciąż po piętach depcze mi.