\section{W portowej knajpie}

Słowa: Eugeniusz Wróblewski\\
Muzyka: Mirosław Peszkowski, Włodzimierz Jabłoński

\index{W portowej knajpie usiadł za stołem}
\vspace{2em}
\begin{tabular}{@{}p{7cm}@{}l@{}}
W portowej knajpie usiadł za stołem  & \bfseries   a D \\
Bosman kulawy i patrzy w tłum.  & \bfseries   a G a \\
Wnet go ferajna otoczy kołem,  & \bfseries   D \\
Gospodarz stawia jamajka rum.  & \bfseries   h E \\
Wszyscy jak swego druha witają,  & \bfseries   a D \\
Co wrócił do nich z dalekich mórz.  & \bfseries   a G a \\
O kraje, lądy, miasta pytają,  & \bfseries   D \\
Czy wiele przeżył sztormów i burz?  & \bfseries   h E \\
\end{tabular}

\vspace{1em}
\begin{tabular}{@{}p{7cm}@{}l@{}}
O, la, hej! Pijmy rum, pijmy rum!  & \bfseries   a G a G a \\
\end{tabular}

\vspace{1em}
\begin{tabular}{@{}p{7cm}@{}l@{}}
Lecz bosman milczy, trunku nie raczy  & \bfseries   e h \\
I wypatruje dziewczyny swej,  & \bfseries   a e \\
Której niestety w gronie słuchaczy  & \bfseries   F B \\
Nie znalazł dzisiaj, ni znaku jej.  & \bfseries   G E7 \\
\end{tabular}

\vspace{1em}
Pochylił bosman głowę nad stołem, \\
Z rumem się miesza opadła łza, \\
A tłum, co zwartym otaczał kołem, \\
Odrzekł - odeszła dziewczyna twa. \\


Podniósł się ciężko, złamany ciosem, \\
Oczami tępo powiódł przez tłum \\
I milcząc wyszedł za swoim losem \\
W stronę, skąd morza wabił go szum. \\
\newpage

I dotąd nie wie żaden z kamratów, \\
Gdzie zniknął stary serdeczny druh. \\
Mówią, że często wraca z zaświatów \\
Zakochanego bosmana duch. \\

Zjawia się w sztormach, w morskiej kipieli \\
I w szumie wiatrów obnosi żal. \\
Ci, co pływają, nieraz widzieli \\
Postać bosmana wśród morskich fal. \\