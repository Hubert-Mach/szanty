\section{Na Mazury}

Słowa i muzyka: Lucjusz Michał Kowalczyk\\

\index{Sie masz, witam Cię, piękną sprawę mam}
\vspace{2em}
\begin{tabular}{@{}p{9cm}@{}l@{}}
Sie masz, witam Cię, piękną sprawę mam, & \bfseries  D A D \\
Pakuj bety swe i leć ze mną tam, & \bfseries  D A D \\
Rzuć kłopotów stos, no nie wykręcaj się, & \bfseries  G D A D \\
Całuj wszystko w nos, & \bfseries  G D \\
Osobowym drugą klasą przejedziemy się. & \bfseries  A \\
\end{tabular}

\vspace{1em}
\begin{tabular}{@{}p{9cm}@{}l@{}}
Ref.: Na Mazury, Mazury, Mazury, & \bfseries  D G D \\
Popływamy tą łajbą z tektury, & \bfseries  D A D \\
Na Mazury, gdzie wiatr zimny wieje, & \bfseries  D G D \\
Gdzie są ryby i grzyby, i knieje. & \bfseries  D A D \\
Tam, gdzie fale nas bujają,  & \bfseries G D \\
Gdzie się ludzie opalają,  & \bfseries A D \\
Wschody słońca piękne są  & \bfseries G D \\
I komary w dupę tną.  & \bfseries A D \\
Gdzie przy ogniu gra muzyka  & \bfseries G D \\
I gorzała w gardłach znika, & \bfseries A D \\
A Pan Leśniczy nie wiadomo skąd & \bfseries G D \\
Woła do nas, do nas, do nas... & \bfseries A D \\
..."Paszła won!"  zaraza żeglarska!  & \bfseries
\end{tabular}

\vspace{1em}

Uszczelniłem dno, lekko chodzi miecz, \\
Zęzy smrodów sto przewietrzyłem precz, \\
Ster nie spada już i grot luzuje się, \\
Więc się ze mną rusz, \\
Już nie będzie tak jak wtedy, nie denerwuj się. \\
\newpage

Skrzynkę piwa mam, Ty otwieracz weź, \\
Napić Ci się dam, tylko mi ją nieś, \\
Coś rozdziawił dziób i masz głupi wzrok? \\
No nie stój jak ten słup, \\
Z Węgorzewa na Ruciane wykonamy skok. \\

ref... x2                                          | (drugi ref. sekundę wyżej) \\
(...) \\
Woła do nas... "Dzień dobry (kochane) żeglarze! Witam! Witam!" \\