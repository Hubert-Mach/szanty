\section{Dziewczyny z Saint Pierre i z Saint Johns}

Słowa i muzyka: Jerzy Porębski

\vspace{2em}
\begin{tabular}{@{}p{10cm}@{}l@{}}
Dobrą lecz krótką pamięć mam, & \bfseries C\\
Więc kiedy to było, no?... Nie wiem sam. \\
Pływaliśmy na Georges Bank, & \bfseries dG\textsuperscript{7}\\
Ławice z piasku, New Foundland. & \bfseries F C\\
\end{tabular}

\vspace{1em}
Śledziowe żniwa, statków tłok, \\
Przy burcie burta, bokiem w bok, \\
Ciężka robota noc i dzień, \\
Wyglądał człek jak szczur, jak cień. \\

\begin{tabular}{@{}p{10cm}@{}l@{}}
Ref.: Marzyło się wtedy, oj marzyło, mon chere, & \bfseries C F C\\
O pięknych, eleganckich panienkach z St. Pierre. & \bfseries d G\textsuperscript{7}\\
Oj, kojarzyło się słówko "I want" & \bfseries FC\\
Z prześliczną, uśmiechniętą panienką z St. Johns. & \bfseries F C D\textsuperscript{7} G\textsuperscript{7} C\\
\end{tabular}

\vspace{1em}
Gdy już przeżarła łapy sól, \\
Załadowano ship na full, \\
Dziób naprowadzał nas na cel, \\
Raz do St. Johns, raz do St. Pierre. \\

A tam, że czasu było w bród, \\
Spokojne życie człowiek wiódł, \\
Języki znałem byle jak: \\
"Je t'aime", "I love you" i... "Would you like?" \\

Ref.: Dziewczyny z St. Johns "Elles n'ont pas peur", \\
Latały awionetką z St. Johns do St. Pierre. \\
Dziewczyny z St. Pierre krzyczały "Go on!" \\
Latając awionetką z St. Pierre do St. Johns. \\

W kabinie czysto, istny raj, \\
Na drzwiach szlafroczki wiszą dwa, \\
Jeden z St. Johns, drugi z St. Pierre, \\
Jeden - "My Dear", drugi - "Mon Chere". \\

Na stole szkocka i chateaubriande, \\
Gatunek do uznania pań, \\
Są fatałaszki, l'eau de Cologne \\
I ta najlepsza w świecie woń... \\

Ref.: Dziewczyny z St. Johns płakały nie raz, \\
Kiedy, po porcie, na morze przyszedł czas. \\
Dziewczyny z St. Pierre żegnały przez łzy, \\
Gdyśmy, po porcie, na morze szli. \\

Kochałem je obie - tę z Johns i tę z Pierre, \\
I starym kawalerem jestem po dziś dzień. \\
Być może, że spotkam je pod koniec moich dni \\
I dalej już we trójkę będziemy razem szli. \\

Ref.: Ta mała z St. Pierre i ta większa z St. Johns \\
I ja razem z nimi, bo kocham je wciąż. \\
Ta mała z St. Pierre i ta nieco większa (trochę grubsza) z St. Johns, \\
I ja razem z nimi - jako ich mąż! \\
