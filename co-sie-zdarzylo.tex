\section{Co się zdarzyło}

Słowa: B. Choiński, Marek Dagnan\\
Muzyka:  trad.

\index{Co się zdarzyło jeden raz}
\vspace{2em}
\begin{tabular}{@{}p{9cm}@{}l@{}}
Co się zdarzyło jeden raz, & \bfseries  D G \\
Nie uwierzycie mi, & \bfseries  A\textsuperscript{7} D \\
Gdym wdepnął sobie w piękny czas & \bfseries  G A\textsuperscript{7} \\
W New-Yorską Chatham Street. & \bfseries  D \\
\end{tabular}

\vspace{1em}
\begin{tabular}{@{}p{9cm}@{}l@{}}
Ref.: Ciągnij pomalutku, aż do skutku, & \bfseries  D G A\textsuperscript{7} D \\
Jak te New York girls, co nie tańczą polki! & \bfseries G A\textsuperscript{7} D \\
\end{tabular}

\vspace{1em}
Dziewuszka grzeczna vis-a-vis obrała kurs na wprost: \\
"Tuż-tuż na Bleeker mieszkam Street." - Więc się zabrałem z nią. \\

"Kielicha" - rzekła - "znajdziesz tam, przekąskę, to i sio." \\
Mamusia nas witała w drzwiach, siostrzyczek cały rząd. \\

Do wierzchu nalewano w szkło i pito aż po dno, \\
Aż zaczął kręcić się, jak bąk, mój globus w miejscu, o! \\

Przyjaciółeczkę wzrusza, bo ja przypominam jej \\
Braciszka, co w Szanghaju jest za nie wiadomo co. \\

Przejrzałem nagle, oczy trąc - Królową Boże chroń! \\
Ten goły facet w łóżku jest bez wątpliwości mną! \\

Pustynną iście zionie w krąg golizną każdy kąt, \\
A bez bielizny, forsy ktoś jest bez wątpienia mną. \\

\newpage
To, w czym do portu szedłem, dość wytworną miało woń \\
Po mączce rybnej wór, ahoj! Jak ulał na Cape Horn! \\

Ref.: Wszystko pomalutku, ciągnie się do skutku, \\
Jak te New York Girls, co nie tańczą... break dance! \\
