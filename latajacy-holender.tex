\section{Latający Holender}

Słowa: Jarosław ("Królik") Zajączkowski\\
Muzyka:  trad.

\index{Jestem duchem oceanów}
\vspace{2em}
\begin{tabular}{@{}p{9cm}@{}l@{}}
Jestem duchem oceanów, & \bfseries  C a \\
Jestem groźny "Flying Dutch". & \bfseries  d C G \\
Na razie straszyć was nie mogę, & \bfseries C a \\
Na razie w porcie muszę stać, & \bfseries  d C G \\
Bo wszystkie wręgi połamane & \bfseries  C a \\
A burty zaczynają gnić. & \bfseries  C G \\
Zdradzieckim sztormem rozszarpana, & \bfseries C a \\
Ostatnia w żaglu pękła nić. & \bfseries d C G \\
\end{tabular}

\vspace{1em}
\begin{tabular}{@{}p{9cm}@{}l@{}}
Ref.: Ragi daga dudam dudam dudam,  | & \bfseries  C a \\
Ragi daga dudam dudam dej.    |bis  & \bfseries d G \\
\end{tabular}

\vspace{1em}
Powrócę wkrótce znów na morze \\
Posępny, siny, dumny, zły. \\
Na kursie sennie się położę, \\
Wyłonię się z porannej mgły. \\
A gdy mnie ujrzysz miły bratku, \\
Spełnię przepowiednię złą: \\
Popłyniesz na swym pięknym statku, \\
Popłyniesz wprost do Bay Hilo. \\

Niedługo znów się zobaczymy, \\
Na morze rzucę blady strach, \\
Z kostuchą razem zatańczymy \\
Białego walca na trzy dwa. \\
Na reje wciągnę sine żagle, \\
Otulę się w zdradzieckie mgły \\
I sennie sztilem kołysany, \\
Po nocach będę wam się śnił.
