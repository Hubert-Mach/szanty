\section{Złota Arka}

Słowa: Dorota Potoręcka\\
Muzyka:  trad.

\index{Hej, kroplę "Krwi Nelsona" każdy chętnie chlapnąłby!}
\vspace{2em}
\begin{tabular}{@{}p{9cm}@{}l@{}}
Hej, kroplę "Krwi Nelsona" każdy chętnie chlapnąłby! & \bfseries  e \\
Hej, kroplę "Krwi Nelsona" każdy chętnie chlapnąłby! & \bfseries  D \\
Hej, kroplę "Krwi Nelsona" każdy chętnie chlapnąłby! & \bfseries  e \\
- I my wszyscy razem z nim! & \bfseries  e D C H\textsuperscript{7} e \\
\end{tabular}

\vspace{1em}
\begin{tabular}{@{}p{9cm}@{}l@{}}
Ref.: Dalej w dół - stara krypo kiwaj się! & \bfseries  e \\
I w górę - złota arko ciągnij się! & \bfseries  D \\
No i znowu w dół - kupo złomu kiwaj się! & \bfseries  e \\
I do domu szybko leć!  & \bfseries  e D C H\textsuperscript{7} e \\
\end{tabular}

\vspace{1em}
Na zagrychę krwisty befsztyk już przed nosem pachnie mi! \\
- I nam wszystkim razem z nim! \\

A usmażyłby mi go wesolutki, tłusty kuk! \\
- I nam wszystkim zaraz też! \\

Pofiglować sobie nieco, zaraz zdrowo byłoby! \\
- I nam wszystkim zaraz też! \\

Zamiast moknąć tu na wachcie, pod pokładem wolę być! \\
- Przydział grogu wolno pić! \\

Całą nockę z kobitkami, szalałbym nawet przez sen! \\
- Lepiej w łóżku w porcie, hen! \\

Zamiast gnić na starej łajbie, lepiej farmę chyba mieć! \\
- I pod gruszą w słonku lec!