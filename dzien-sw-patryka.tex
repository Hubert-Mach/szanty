\section{Dzień św. Patryka}

Słowa: Sławomir Klupś \\
Muzyka:  trad.

\index{Gdy w Nowym Jorku wstawał świt}
\vspace{2em}
\begin{tabular}{@{}p{9cm}@{}l@{}}
Gdy w Nowym Jorku wstawał świt, & \bfseries G C \\
Na morzu szalał sztorm. & \bfseries G D G \\
Sam diabeł piekieł otworzył drzwi & \bfseries G D G \\
I ciągnął ship nasz na dno. & \bfseries G D \\
Już wściekłe fale wyrwały maszt, & \bfseries G D G \\
Nadziei zniknął już cień. & \bfseries G D \\
\end{tabular}

\vspace{1em}
\begin{tabular}{@{}p{9cm}@{}l@{}}
Ref.: To Patryka dzień, Patryka dzień, & \bfseries C G C \\
To był Patryka dzień. & \bfseries G D G \\
\end{tabular}

\vspace{1em}
Już każdy widzi, że to kres \\
I z życiem żegna się. \\
Kapitan nawet, ten podły łotr, \\
Do Świętych modlić się chce. \\
Uratuj Panie, patronie nasz, \\
A czcić będziemy ten dzień. \\
\begin{multicols}{2}
Już na nic prośby, na nic łzy, \\
Pochłonie żywioł nas. \\
Za podłe życie, parszywy los, \\
Rachunki spłacić już czas. \\
W spienionych falach szedł na dno ship \\
I kończył życia nasz dzień. \\

Z załogi tylko pięciu z nas \\
Ujrzało stały ląd. \\
Za całą forsę w kościele swym \\
Kupiliśmy nowy dzwon. \\
I bije on, kiedy morze grzmi \\
I chwalić będzie ten dzień.
\end{multicols}
