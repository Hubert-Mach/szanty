\section{Bitwa}

Słowa i muzyka: Lech Klupś \\

\index{Okręt nasz wpłynął w mgłę}
\vspace{2em}
\begin{tabular}{@{}p{10cm}@{}l@{}}
Okręt nasz wpłynął w mgłę i fregaty dwie & \bfseries  e D C a \\
Popłynęły naszym kursem by nie zgubić się. & \bfseries  e D G H \\
Potem szkwał wypchnął nas poza mleczny pas & \bfseries  e D C a \\
I nikt wtedy nie przypuszczał, że fregaty śmierć nam niosą. & \bfseries  e D G H \\
\end{tabular}

\vspace{1em}
\begin{tabular}{@{}p{10cm}@{}l@{}}
Ref.: Ciepła krew poleje się strugami, & \bfseries  G D e h \\
Wygra ten, kto utrzyma ship. & \bfseries  C D e \\
W huku dział ktoś przykryje się falami, & \bfseries G D e h \\
Jak da Bóg, ocalimy bryg. & \bfseries  C D e \\
\end{tabular}

\vspace{1em}
Nagły huk w uszach grał i już atak trwał, \\
To fregaty uzbrojone rzędem w setkę dział. \\
Czarny dym spowił nas, przyszedł śmierci czas, \\
Krzyk i lament mych kamratów, przerywany ogniem katów. \\

\vspace{-1em}
Pocisk nasz trafił w maszt, usłyszałem trzask, \\
To sterburtę rozwaliła jedna z naszych salw. \\
"Żagiel staw" krzyknął ktoś, znów piratów złość, \\
Bo od rufy nam powiało, a piratom w mordę wiało. \\

Z fregat dwóch tylko ta pierwsza w pogoń szła, \\
Wnet abordaż rozpoczęli, gdy dopadli nas. \\
Szyper ich dziury dwie zrobił w swoim dnie, \\
Nie pomogło to psubratom, reszta z rei zwisa za to. \\

Po dziś dzień tamtą mgłę i fregaty dwie, \\
Kiedy noc zamyka oczy, widzę w moim śnie. \\
Tamci, co śpią na dnie, uśmiechają się, \\
Że ich straszną śmierć pomścili bracia, którzy zwyciężyli.