\section{Na Cytryniarzu}

Słowa: Jerzy Rogacki\\
Muzyka:  trad.

\index{Jeżeli chcesz na statku tym popływać parę lat}
\vspace{2em}
\begin{tabular}{@{}p{10cm}@{}l@{}}
Jeżeli chcesz na statku tym popływać parę lat,             & \bfseries   C G\textsuperscript{7} C \\
Nie będziesz miał kłopotów, gdy swe obowiązki znasz.       & \bfseries   C F C \\
O prawa twe zadbano też, ministra podpis jest,             & \bfseries   C F C \\
U nas na cytryniarzu wszystko gra - ustawa święta rzecz.   & \bfseries   C G\textsuperscript{7} C \\
\end{tabular}

\vspace{1em}
\begin{tabular}{@{}p{10cm}@{}l@{}}
Ref.: Wybierać grota nawietrzny bras, a na zawietrznym luz!   & \bfseries   C G\textsuperscript{7} C \\
Stawiamy szmaty, do fałów, hej! Zaczyna dmuchać już.    & \bfseries   C F C \\
Hurra! Hej hurra! Śpiewamy refren ten.                  & \bfseries   C\textsuperscript{7} F C \\
Tylko u nas na cytryniarzu tak wspaniale pływa się!     & \bfseries   C G\textsuperscript{7} C \\
\end{tabular}

\vspace{1em}
Gdy już na burcie będziesz, to przeczytasz wszystko sam. \\
Ile zjesz chleba, masła, mięsa, jest napisane tam, \\
Herbaty, kawy, cukru, octu, soku z limonów też. \\
U nas na cytryniarzu wszystko gra - ustawa święta rzecz. \\

Wachta za wachtą, dzień za dniem, rozkład na pamięć znasz. \\
Dziesięć dni minęło gdzieś, na sok z limonów czas. \\
Zabierać brudne łapy precz! Do fałów ruszać się! \\
Teraz stawiać będziemy grota, a komenda święta rzecz. \\

I znowu wachta, i znów na dek. A niech to trafi szlag! \\
Po paru godzinach będzie zmiana - myślisz sobie tak. \\
Wybiła szklanka, nareszcie czas by w koję zwalić się. \\
U nas na cytryniarzu wszystko gra, a ustawa święta rzecz. \\