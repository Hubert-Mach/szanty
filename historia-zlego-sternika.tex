\section{Historia złego sternika}

Słowa: Radosław Jóźwiak\\
Muzyka: Artur Chudź

\index{Posłuchajcie opowieści}
\vspace{2em}
\begin{tabular}{@{}p{7cm}@{}l@{}}
Posłuchajcie opowieści,  & \bfseries   e D e \\
- Hej, heja, ho!  & \bfseries   e D e \\
Co ładunek grozy mieści,  & \bfseries   e D G D \\
- Hej, heja, ho!  & \bfseries   e D e \\
Której strasznym bohaterem  & \bfseries   e D e \\
- Hej, heja, ho!  & \bfseries   e D e \\
Jest, stojący tam za sterem,  & \bfseries   e D e D \\
Bardzo wredny łysy typ,  & \bfseries   C G \\
Co ma w szachu cały ship.  & \bfseries   a H\textsuperscript{7} e \\
\end{tabular}

\vspace{1em}
\begin{tabular}{@{}p{7cm}@{}l@{}}
Ref.: Dalej, chłopcy, ciągnąć fały  & \bfseries   G D \\
Pod nadzorem łysej pały.  & \bfseries   G D \\
Ciągnij, choćbyś padł na pysk,  & \bfseries   e h \\
Gdy nad Tobą jaja błysk.  & \bfseries   a H\textsuperscript{7} e \\
\end{tabular}

\vspace{1em}
\begin{tabular}{@{}p{7cm}@{}l@{}}
Brać się, chłopcy, brać do pracy,  & \bfseries   G D \\
Pod nadzorem wrednej glacy.  & \bfseries   G D \\
Może znajdzie się na dnie  & \bfseries   e h \\
Typ, co gąbką czesze się.  & \bfseries   a H\textsuperscript{7} e \\
\end{tabular}

\vspace{1em}
W porcie, gdzieś na Bliskim Wschodzie, \\
Gdzie łysina nie jest w modzie, \\
Powiedziały cud hurysy, \\
Jaki brzydki jest ten łysy, \\
Choćby górę złota dał, \\
Nie skorzysta z naszych ciał. \\
\newpage

Tak okrutny bywa los, \\
Mu nie spadnie z głowy włos. \\
Z desperacji więc matrosy, \\
Wyrywają sobie włosy, \\
Bo ten widok takich glac, \\
Męczy bardziej niźli kac. \\

Kiedyś, jakiś inny bryg, \\
Na dnie morza zniknął w mig, \\
Bo te blaski łysej pały, \\
Mu latarnią się wydały \\
I gdy kurs swój zmienić chciał, \\
To roztrzaskał się wśród skał. \\