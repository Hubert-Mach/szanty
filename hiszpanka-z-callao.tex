\section{Hiszpanka z Callao}

Słowa: Sławomir Klupś, Henryk ("Szkot") Czekała\\
Muzyka:  trad.

\index{Przepuściłem znów całą forsę swą}
\vspace{2em}
\begin{tabular}{@{}p{7cm}@{}l@{}}
Przepuściłem znów całą forsę swą  & \bfseries   G e \\
Na hiszpańską dziewkę z Callao.  & \bfseries   a G D \\
Wyciągnęła ze mnie cały szmal  & \bfseries   G e \\
I spłukałem na nią się do cna.  & \bfseries   a G D \\
\end{tabular}

\vspace{1em}
\begin{tabular}{@{}p{7cm}@{}l@{}}
Już znikają główki portu,  & \bfseries   G e \\
W którym stary został dom.  & \bfseries   G D \\
Znów za rufą niknie w dali  & \bfseries   G e \\
Ukochany stały ląd.  & \bfseries   a G D \\
\end{tabular}

\vspace{1em}
\begin{tabular}{@{}p{7cm}@{}l@{}}
Ref.: Dalej chłopcy, rwijmy fały,  & \bfseries   G e \\
Niech uderzy w żagle wiatr.  & \bfseries   a G D \\
W morze znowu wypływamy -  & \bfseries   G e \\
Ile tam spędzimy lat?  & \bfseries   a G D \\
\end{tabular}

\begin{multicols}{2}
Już niejeden pokład mym domem był \\
I niejeden kot me plecy bił. \\
Choć robota ciężka i żarcie psie, \\
W Callao po rejsie znajdziesz mnie. \\

Znów znikają główki portu, \\
W którym stary został dom. \\
Znów za rufą niknie w dali \\
Ukochany stały ląd. \\

Zaprószyłem tam nieraz głowę swą, \\
W barze u Hiszpanki z Callao. \\
Butlę rumu wielką dała mi \\
I ciągnęła ze mną aż po świt. \\

Już znikają główki portu, \\
W którym stary został dom. \\
Znów za rufą niknie w dali \\
Ukochany stały ląd. \\

A pod Cape Horn, gdzie nas poniesie wiatr, \\
Popłyniemy z sztormem za pan brat. \\
Przez groźne burze, śnieg i lód \\
Dojdziemy aż do piekła wrót. \\

Choć grabieją z zimna ręce, \\
Zadek mokry dzień i noc, \\
Jedna myśl rozgrzewa serce  \\
O Hiszpance z Callao!
\end{multicols}