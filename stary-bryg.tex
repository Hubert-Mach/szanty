\section{Stary bryg (Statek widmo)}

Słowa: W. Skalski\\
Muzyka: Krzysztof Klenczon

\index{Gdy wypływał z portu stary bryg}
\vspace{2em}
\begin{tabular}{@{}p{9cm}@{}l@{}}
Gdy wypływał z portu stary bryg,  <Hej! Stary bryg!> & \bfseries  e D G D e <D e> \\
Jego losów nie znał wtedy nikt.   <Wtedy nikt! Wtedy nikt!> & \bfseries e D G D <G H7> \\
Nikt nie wiedział o tym, że & \bfseries  e D \\
Statkiem-widmem stanie się stary bryg. & \bfseries  A H7 e \\
\end{tabular}

\vspace{1em}
\begin{tabular}{@{}p{9cm}@{}l@{}}
Ref.: Hej, ho! Na "Umrzyka Skrzyni" & \bfseries  e G D e \\
I butelka rumu.  <Rumu!> & \bfseries G D e <D e>  \\
Hej, ho! Resztę czart uczyni \\
I butelka rumu.  <Rumu!> \\
\end{tabular}

\vspace{1em}
Co z załogą zrobił stary bryg,    <Hej! Stary bryg!> \\
Tego też nie zgadnie chyba nikt.  <Chyba nikt! Chyba nikt!> \\
Czy zostawił w porcie ją, \\
Czy na morza dnie? - Nikt nie wie gdzie. \\

Przepowiednia zła jest, że ho ho!  <Hej! Że ho ho!> \\
Kto go spotka - marny jego los.    <Jego los! Jego los!> \\
Ale my nie martwmy się, \\
Hej! Nie martwmy się - rum jeszcze jest. \\
